%%
%% Toolbox Language Manual
%% $Id: DLMan.tex,v 1.11 2006/04/19 10:24:48 vdmtools Exp $
%% 

%%%%%%%%%%%%%%%%%%%%%%%%%%%%%%%%%%%%%%%%
% PDF compatibility code. 

\makeatletter
\newif\ifpdflatex@
\ifx\pdftexversion\@undefined
\pdflatex@false
%\message{Not using pdf}
\else
\pdflatex@true
%\message{Using pdf}
\fi

\newcommand{\latexpdf}[2]{
  \ifpdflatex@ #1
  \else #2
  \fi
}

\newcommand{\latexorpdf}[2]{
  \ifpdflatex@ #2
  \else #1
  \fi
}

\makeatother

#ifdef A4Format
\newcommand{\pformat}{a4paper}
#endif A4Format
#ifdef LetterFormat
\newcommand{\pformat}{letterpaper}
#endif LetterFormat

%%%%%%%%%%%%%%%%%%%%%%%%%%%%%%%%%%%%%%%%

\latexorpdf{
\documentclass[\pformat,12pt]{article}
}{
% pdftex option is used by graphic[sx],hyperref,toolbox.sty
\documentclass[\pformat,pdftex,12pt]{article}
}

\usepackage{toolbox}
\usepackage{vdmsl-2e}
\usepackage{makeidx}
\usepackage{alltt}
\usepackage{verbatim}

% Ueki change start
\usepackage[dvipdfm,bookmarks=true,bookmarksnumbered=true,colorlinks,plainpages=true]{hyperref}
% Ueki change end

% Ueki delete start
%\latexorpdf{
%\usepackage[plainpages=true,colorlinks,linkcolor=black,citecolor=black,pagecolor=black, urlcolor=black]{hyperref}
%}{
%\usepackage[plainpages=true,colorlinks]{hyperref}
%}
% Ueki delete end

\newcommand{\meti}[1]{\item[#1]\mbox{}\\}
\newcommand{\Lit}[1]{`#1\Quote}
\newcommand{\Rule}[2]{
  \begin{quote}\begin{tabbing}
    #1\index{#1}\ \ \= = \ \ \= #2  ; %    Adds production rule to index
  \end{tabbing}\end{quote}
  }

\newcommand{\SeqPt}[1]{\{\ #1\ \}}
\newcommand{\lfeed}{\\ \> \>}
\newcommand{\OptPt}[1]{[\ #1\ ]}
\newcommand{\dsepl}{\ $|$\ }
\newcommand{\dsep}{\\ \> $|$ \>}
\newcommand{\Lop}[1]{\Lit{\kw{#1}}}
\newcommand{\Sig}[1]{\Lit{{\tt #1}}}
\newcommand{\blankline}{\vspace{\baselineskip}}
\newcommand{\Brack}[1]{(\ #1\ )}
\newcommand{\nmk}{\footnotemark}
\newcommand{\ntext}[1]{\footnotetext{{\bf Note: } #1}}


%\usepackage[dvips]{color}
\usepackage{longtable}
\definecolor{covered}{rgb}{0,0,0}     %black
\definecolor{not_covered}{gray}{0.5} %gray

\parindent0mm

\newlength{\keywwidth}

\newcommand{\xfigpicture}[4]{
\begin{figure}[hbt]
\setlength{\unitlength}{1mm}
\begin{center}
\mbox{
\begin{picture}(#1,#2)
\put(0,0){\special{psfile=#3 hscale=70 vscale=55}}
\end{picture} }
\end{center}
\caption{#4}
\end{figure}
}


\newcommand{\vdmslpp}{VDM-SL}
\newcommand{\Toolbox}{Toolbox}
\newcommand{\aaa}{\tt }
\newcommand{\cmd}{\tt }
\newcommand{\id}[1]{%
\settowidth{\keywwidth}{\tt #1}%
\protect\makebox[\keywwidth][l]{{\it #1}}}
\nolinenumbering

\begin{document}
\vdmtoolsmanualscsk{The Dynamic Link Facility}{2.0}


\section{Introduction}

This manual describes a feature of the VDMTools called the
Dynamic Link facility. This feature enables that \vdmslpp\ 
specifications can be combined with code written in C++. This
combination enables that a \vdmslpp\ specification during
interpretation can use and execute parts that are written in C++. It
is called the Dynamic Link facility because the C++ code is
dynamically linked together with the Toolbox.

Use of this combination can be of interest in cases where only part of
a system is developed using \vdmslpp. It can be valuable if:

\begin{itemize}
\item a new component should be developed using VDM-SL for a system
  which already has been implemented;
\item a system being specified in VDM-SL needs to make use of features
  which are not available in VDM-SL itself; and
\item it is desirable only to develop a part of the system using
  VDM-SL.
\end{itemize}

With this Dynamic Link facility it is possible to investigate the
interaction of the parts specified in VDM-SL with parts implemented in
C++ at an early stage in the development process.


\subsection{Using This Manual}

This document is an extension to the {\it User Manual for the
 VDM-SL Toolbox} \cite{UserMan-SCSK}. Before continuing reading this
document it is recommended to read the {\it Dynamic Link Modules}
section of {\it The VDM-SL Language} \cite{LangMan-SCSK}. Knowledge
about C++ \cite{Stroustrup91} and {\it The VDM C++ Library}
\cite{LibMan-SCSK} is also assumed.

You do not need to read this entire manual to get started using the
Dynamic Link feature. Start by reading Section~\ref{getting-started}
to get a detailed description of an example using this feature. All
files from this example are also present in the appendices.
Section~\ref{sec:dlcomponents} describes the components which are
involved when using the Dynamic Link facility one by one. Those who
are interested in the way this feature has been implemented internally
in the Toolbox should consult \cite{Frohlich&96}.

Appendix~\ref{sec:sysreq} provides detailed information about system
requirements.


\section{Getting Started}
\label{getting-started}

The combination of formal specifications and code written in C++ only
becomes feasible by establishing a common framework combining them.
The most basic problem is that the values in these two different
worlds have different representations. For the two worlds to
communicate, it is necessary to convert values from the specification
world to the code world and vice versa.  By means of small examples
this section illustrates the idea of integrating code with a
specification in order to obtain the required functionality.

The examples present the integration of trigonometric functions into a
specification using the VDM-SL Toolbox.  Trigonometric functions
are not included in VDM-SL so in order to develop a system which needs
such functions the Dynamic Link facility is used.

\subsection{The Basic Idea}

The aim of this approach is to be able to analyse the combination of
specification and implementation. By combining code and specification
the main intention is to provide a prototyping facility by integrating
the execution of code into the interpretation of a specification.  We
enable definitions, made at the code level, to be integrated with a
formal specification, such that interpretation of a specification code
definitions can be executed.  Therefore we distinguish between the
specification level, which refers to the \vdmslpp\ specification and
integration parts made at within the specification, and the code
level, which relates to the code and its integration for the
execution.

Figure~\ref{idea} shows the components involved in combining
specification and code. Firstly there are the \vdmslpp\ specification
part and the external C++ code part, these are the light grey boxes in
the figure. Secondly there is the part that interface the
specification and code parts, this is the dark grey boxes in the
figure. This interface consists of a part at specification level
(called {\em Dynamic Link modules\/} or simply {\em DL modules\/}) and
a part at code level (called {\em type conversion functions\/})

 
\begin{figure}
\begin{center}
\resizebox{.55\textwidth}{!}%
{\includegraphics{approach}}
\caption{Combination of code and VDM-SL specification\label{idea}}
\end{center}
\end{figure}

A DL module describes the \vdmslpp\ interface for every definition of
the C++ code that will be used in the VDM-SL specification. This
information is described as \vdmslpp\ function, operation, and value
definitions.

As the C++ code and the VDM-SL Toolbox have different
representations for their values, type conversion functions to
transform between the tow representations must be defined. These must
be defined at the code level.  A number of conventions for these type
conversion functions have been defined.  Generally, the values that
are exchanged are of types from the VDM C++ Library. (The VDM C++
library contains classes implementing all the VDM types such as Map,
Set, Sequence, Char, Bool, etc.  This library is described in detail
in \cite{LibMan-SCSK} and defined in the library file {\tt libvdm.a}, which
is included in the \Toolbox\ distribution.)

The convention for calling a C++ function taking parameters and
returning a value is as follows:
\begin{itemize}
\item The parameters are passed as one value of type ``Sequence'' from
  the VDM C++ Library, with each parameter being an element of the
  sequence: the first argument being the first element, the second
  argument being the second element etc.
\item The returned value is of type Generic (which any VDM Library
  class type can be converted to).
\end{itemize}

For calling C++ values and constants a function converting and
returning the value/constant must be defined.

Beside that type conversion functions to convert between the VDM C++
values and the C++ code must be defined, no modifications of the C++
code need to be performed.

 
\subsection{An Example with Trigonometric Functions}

In order to illustrate the idea an example which integrates 
trigonometric functions into a specification is presented.

The C++ code, which is to be integrated into a VDM-SL specification,
is  given by the mathematical standard C library called $math$.
The examples make use of implementations of the trigonometric functions  
$sin$, $cos$ and of the constant $pi$. First, a specification
which applies these definitions is presented and then the extensions 
at specification level as well as at code level are shown.

\subsection*{The VDM-SL Specification}

The following VDM-SL specification gives an example of how the
definitions of the DL module {\tt MATHLIB} are imported into the module
{\tt CYLINDER}. Notice how the import from a DL module is identical to
the import from an ordinary module, i.e.\ the module importing does
not know whether the imported module is an ordinary module or a DL
module.

\begin{quote}
\begin{verbatim}
module CYLINDER

imports
  from MATHLIB
    functions
      ExtCos : real -> real;
      ExtSin : real -> real
   
    values
      ExtPI : real

 definitions
   functions
     CircCyl_Vol : real * real * real -> real
     CircCyl_Vol (r, h, a) ==
       MATHLIB`ExtPI * r * r * h * MATHLIB`ExtSin(a)

end CYLINDER      
\end{verbatim}
\end{quote}

The module {\tt CYLINDER} imports the functions {\tt ExtCos} and 
{\tt ExtSin} and the value {\tt ExtPI} from the DL
module {\tt MATHLIB}.
The function {\tt CircCyl\_Vol} evaluates the volume of a circular cylinder
and makes use of the constant {\tt ExtPI} as well as the function {\tt ExtSin}.
 
\subsection*{The Interface at VDM-SL Level}

As mentioned above, the interface between code and specification has
to be provided at two different levels. The interface at the VDM-SL
level declares the VDM-SL types of the functions that are exported to
the outside world.  The interface at the code level is based on the
definition of the above mentioned type conversion functions.

At the VDM-SL level the interface looks like:

\begin{quote}
\begin{verbatim}
dlmodule MATHLIB

  exports
    functions
      ExtCos : real -> real;
      ExtSin : real -> real   
      
   values   
      ExtPI : real
      
   uselib
    "libmath.so"

end MATHLIB
\end{verbatim}
\end{quote}

A DL module must always contain an export section which declares all
constructs that are available to the outside world.  The constructs
from the export section can be imported by other modules.  The export
section consists of the signatures for function and operation
definitions and the type information for value definitions defining
the \vdmslpp\ interface to the C++ code.  A value declaration in a DL
module relates either to a constant or a variable definition in the
code.

The {\tt uselib} field contains the name of the shared object file,
which contains the code accessed through the DL module.


\subsection*{The Interface at C++ Level}

The interface at the code level is developed in C++ and consists of a
number of declarations and the definitions of the type conversion
functions. The declaration part includes the standard mathematical
library.  The type conversion functions convert the VDM C++ sequence
value used by the Toolbox to values accepted by the C++ code and vice
versa.  A generic VDM C++ type can have an underlying value of any VDM
type (e.g.\ the generic VDM C++ type {\tt Sequence} represents a
VDM-SL sequence which can contain arbitrary VDM elements.)

The interface looks like:

\begin{quote}
\begin{verbatim}
#include "metaiv.h"
#include <math.h>

# Platform specific directives/includes...

extern "C" {
  DLLFUN void InitDLModule(bool init);
  DLLFUN void ExtCos(const Sequence & sq, Generic res);
  DLLFUN void ExtSin(const Sequence & sq, Generic res);
  DLLFUN void ExtPI (Generic res);
}

void InitDLModule(bool init)
{
  // This function is called by the Toolbox when modules are
  // initialised and before they are unloaded.
  // init is true on after load and false before unload. 
}

void ExtCos(const Sequence & sq, Generic res)
{
  res = (Real( cos(Real(sq[1])) ));
}

void ExtSin(const Sequence & sq, Generic res)
{
  res = (Real( sin(Real(sq[1])) ));
}

void ExtPI (Generic res)
{
  res = (Real(M_PI));
}  
\end{verbatim}
\end{quote}

Notice that {\em ExtPI\/} is defined as a \vdmslpp\ value in the
\vdmslpp\ interface, but as a function in the code level interface.

The Toolbox puts the arguments of the called function into a value of
the generic VDM C++ type Sequence and passes it to the type conversion
function. The type conversion function extracts and converts the
elements of the sequence into values required by the integrated C++
code. For example, the type conversion function {\tt ExtSin} extracts
the first element of the sequence and casts it to the VDM C++ type
{\tt Real}, which is casted automatically to the C++ type {\tt
  double}.  This is given as argument to the {\tt sin} function at C++
level and the result is converted into a VDM C++ value and returned to
the \Toolbox.

In this case a C standard library was used as code, but it could as
well have been a user-defined C++ package. The approach is open to any
kind of module developed in C++. The user is only required to develop
a DL module and to define type conversion functions. Note that the
type conversion functions must be enclosed in an {\tt extern "C"}
linkage specification such that C++ parameter type mangling is not
made part of the function symbol name that is stored in the shared
object library.

\subsection*{Creating A Shared Library}

In order to create a dynamically linked library it is required that
the type conversion functions are enclosed in an {\tt extern "C"}
linkage specification as above. In addition, it is required that the
{\tt metaiv.h} header file is included as above. Then the C++ code
must be compiled with one of the supported compilers (see section
\ref{sec:sysreq}), using the necessary flags to generate a shared
library. It is recommended to record the way the library can be 
created in a Makefile; example Makefiles showing the requisite flags
are shown in Appendix~\ref{makefiles}. 
For Solaris 10 this example would have a Makefile
such as\footnote{Under the assumption that the interface at the C++
  level is placed in a file called {\tt tcfmath.cc}.}:

\begin{quote}
\begin{verbatim}
all: libcylio.so libmath.so

%.so:
ifeq ($(strip $(OSTYPE)),Darwin)
        $(CXX) -dynamiclib -fPIC -o $@ $^ $(LIB)
else
        $(CXX) -shared -fPIC -o $@ $^ $(LIB)
endif

libcylio.so: cylio.o tcfcylio.o 

cylio.o: cylio.cc
        ${CXX} -c -fPIC -o $@ $< ${INCL}

tcfcylio.o: tcfcylio.cc
        ${CXX} -c -fPIC -o $@ $< ${INCL}

libmath.so: tcfmath.o

tcfmath.o: tcfmath.cc
        ${CXX} -c -fPIC -o $@  $<  ${INCL}
\end{verbatim}
\end{quote}

The options to be used under other supported platforms/compilers can
be seen in Section~\ref{sec:create}.

\subsection*{Running the Example}

Having created the shared library {\tt libmath.so} it is now possible
to start the Toolbox. The {\tt CYLINDER} module and the {\tt MATHLIB} DL
modules can be read into the \Toolbox\ as any other module by syntax
checking them. Initialising the specification now will establish the
link to the shared library such that it is possible to interpret the
{\tt CircCyl\_Vol} function, where the calls to {\tt MATHLIB`ExtPI} and
{\tt MATHLIB`ExtSin} are calls to the library.  The \vdmslpp\ parts can
also be debugged using the \VDMTools\ debugger, whereas the code parts
for obvious reasons cannot.



\subsection*{Extending this Example}

In the appendices of this document, this small example has been
extended with one more DL module. This extra module illustrates how to
use a simple I/O interface for the input of the dimensions of a
circular cylinder. Other examples from the Examples Repository
(http://www.csr.ncl.ac.uk/vdm/examples/examples.html)
illustrate the use of a more general user interface using Tcl/Tk.
 
\section{Dynamic Link Components}
\label{sec:dlcomponents}

This section describes the different components which are involved
when you are using the Dynamic Link facility, one by one. The
intension is to enable you to use this section as a kind of reference
guide after having read the previous section once.

In this document we have used a general name convention. We would
recommend you to make a convention yourself. It makes it much easier
to find out where constructs are defined. The name convention we has
used have a ``{\tt Ext}'' prefix for all constructs which are defined
externally (i.e.\ in the C++ code). All files with conversion functions
have a ``{\tt tcf}'' prefix.  Finally, on Unix all dynamically linked libraries
have a ``{\tt lib}'' prefix and they all have an {\tt .so} extension;
on Windows all dynamically linked libraries have a \texttt{.dll}
extension. 

\subsection{Dynamic Link Modules}

A {\em DL module} provides the \vdmslpp\ interface to the code parts.
Beside the module name it contains an import section, an export
section, and the name of the library (containing the C++ code) to link
dynamically with.
\begin{itemize}
\item The import section, which is optional, describes \vdmslpp\ types
  used within the DL module.  An example of this is provided in the
  {\tt CYLIO} DL module of Appendix~\ref{sec:math-cylio}.
\item The export section describes the functions, operation and values
  available from the DL module. Notice that unlike ordinary modules in
  DL modules all exported constructs must be explicitely described,
  i.e.\ no {\tt exports all} exists.
\item The name of the shared library describes which library to link
  dynamically with. This can be left our, in which case the
  specification can by syntax and type checked, but not initialised or
  interpreted.  
  
  Section~\ref{sec:uselibpath} describes how to define where the
  \Toolbox\ should search for the library.
\end{itemize}
The precise syntax and semantics of the DL modules is described in
more detail in the {\em Dynamic Link Modules} section of {\em The 
VDM-SL Language} \cite{LangMan-SCSK}.

The difference between functions and operations is the same for DL
modules as for ordinary modules: functions must return a value and
must not have any side effects. In contrast, operations may have side
effects, and need not return any values. Naturally, this is simply a
pragmatic difference which the Toolbox cannot check.


\subsection{Type Conversion Functions}

The purpose of the type conversion functions is to convert the values
used by the Toolbox (these are objects of the {\it VDM C++ Library})
into values required by the integrated code and vice versa.  A type
conversion function must always take as argument an object of the {\it
  VDM C++ Class} type {\tt Sequence} in order to handle functions with
different arity. It returns either an object of the {\it VDM C++
  Class} {\tt Generic} or {\tt void}. A type conversion function must
be defined for each construct that is exported from a Dynamic Link
module.

If you wish to integrate a function which is defined in C++ code, (e.g.\
{\tt double} {\tt Volume (double radius, double height)}),
a type conversion function must be defined.
In this case, the DL module should contain a corresponding function or operation
definition, say {\tt functions ExtVolume: real * real -> real}.
The Toolbox puts the evaluated arguments, given by the interpreted 
specification, into an object of the class {\tt Sequence}, and passes it to 
the type conversion functions. The definition of this small example is:

\begin{quote}
\begin{verbatim}
void ExtVolume (const Sequence & sq1, Generic res)
{
  double rad, height;
  rad = (Real) sq[1];
  height = (Real) sq[2];
  res = ( (Real) Volume(rad, height) );
}
\end{verbatim}
\end{quote}

The arguments for the function {\tt Volume} are extracted 
from the sequence and converted into double values. The return value from
{\tt Volume}
is converted into an object of the class {\tt Real} which is automatically
casted into a {\tt Generic}.

Table~\ref{table1} provides an overview of related definitions in DL
modules, type conversion functions, and C++ code, i.e.\ how values are
represented at the different levels.  Appendix \ref{example} contains
the type conversion functions for the trigonometric example.

\begin{table}
%\rotatebox{90}
\begin{center}
\begin{sideways}
\begin{tabular}{|p{7cm}|p{7cm}|p{6cm}|}
\hline
{\it DL module definition}&{\it type conversion
  function}&{\it C++ code}\\
\hline \hline
value definition&function with no argument and result type {\tt
  Generic}&variable or constant \\ \hline
\verb+values+            & & \\
\verb+  ExtLength: real+ & \verb+Generic ExtLength ()+ & \verb+double Length = 5.0;+ \\
\verb+  ExtMax:nat+      & \verb+Generic ExtMax ()+    & \verb+const int max = 20;+ \\ \hline\hline
function or operation definition & 
function with argument {\tt Sequence} and result {\tt Generic} & 
function with result different from {\tt void} \\ \hline
\verb+functions+ & \verb+Generic ExtVolume(Sequence sq1)+ & \verb+double Volume(double radius,+ \\
\verb+  ExtVolume: real * real -> real+ &                 & \verb+              double height)+ \\
or & & \\
\verb+operations+ & & \\
\verb+  ExtVolume: real * real ==> real+ & & \\ \hline \hline
operation without result 
& function with argument {\tt Sequence} and result type {\tt void}&
function with result {\tt void} \\ \hline
\verb+operations+ & \verb+void ExtShowItem(Sequence sq1)+ & \verb+void ShowItem (int item)+ \\
\verb+  ExtShowItem: nat ==> ()+ & & \\[1ex]
\hline \hline
\end{tabular}
\end{sideways}
\end{center}
\caption{Value representations at different levels.} 
\label{table1}
\end{table}

\subsection{Converting Records}

Special attention needs to be given to records. In the \vdmslpp\ 
specification records are tagged with a name, whereas in the VDM C++
Library they are tagged with an integer. These tags are refered to as
``symbolic names'' and ``numeric tags'' respectively in the remaining
of this section.

The VDM C++ Library can maintain a map (called the Record Info Map),
mapping numeric tags to symbolic names.  Any symbolic name which is
sent to or returned from the code part of a dynamically linked module
must have been registered in this map.

This registering must be defined in a function named {\tt
  InitDLModule}.  If this function exists the \Toolbox\ will call it
during initialisation (when executing the {\tt init} command).  The
example below illustrates how the Record Info Map is built within the
InitDLModule function:
\begin{itemize}
\item For each symbolic name used, a corresponding numeric tag must be
  added in the Record Info Map.  This is performed by executing:

  \hspace*{1em}{\tt VDMGetDefaultRecInfoMap().NewTag(numtag, recordsize);}
  
  where {\tt numtag} is the numeric tag, and {\tt recordsize} is the
  number of fields in the record.
\item The symbolic name must be bound to the numeric tag.  This is
  performed by executing:

  \hspace*{1em}{\tt VDMGetDefaultRecInfoMap().SetSymTag(numtag, "symbname"); }
  
  where {\tt numtag} still is the numeric tag, and {\tt symbname} is
  the symbolic name. Noce that the symbolic name must be encapsulated
  in quotes.
\end{itemize}


In the example the numeric tag and record size are defined as
constants ({\tt TagA\_X} and {\tt TagA\_X\_Size}).

The examples:
\begin{quote}
\begin{verbatim}
extern "C" void InitDLModule(bool init); 
const int TagA_X = 1; 
const int TagA_X_Size = 2; 
void InitDLModule(bool init) {
  if (init) {
    VDMGetDefaultRecInfoMap().NewTag(TagA_X, TagA_X_Size);
    VDMGetDefaultRecInfoMap().SetSymTag(TagA_X, "A`X"); 
  }
}
\end{verbatim}
\end{quote}

When the \Toolbox\ is creating a Record value with name {\tt A`X} to
be sent to the C++ code it will find the numeric tag corresponding to
the sumbolic name {\tt A`X} in the Record Info Map of the DL Module.

Notice the following issues when creating a Record Info Map:
\begin{itemize}
\item The symbolic name of the {\tt SetSymTag} command must be an
  existing record name qualified with the corresponding module name,
  and the record size of the {\tt NewTag} command must be the exact
  number of fields of the record.
\item Each symbolic tag must exists only once in the Record Info Map.
\end{itemize}


\subsubsection{Using Records in combination with the Code Generator}

The Code Generator creates code that sets up the Record Info Map for
code generated modules.  Thus if generated code is compiling and
linking into a dynamically linked library, the InitDLModule function can
simply call the function {\tt init\_\textit{Module}()}, to define the
records in \textit{Module} in the Record Info Map.

%
%\subsection{Converting Tokens}
%
%The token type is exported from the toolbox as a C++ class {\tt Record}.
%However, the tag of token records is always equal to {\tt TOKEN},
%which is a macro declared in the file {\tt cg\_aux.h}, and the number
%of fields in token records is always equal to 1. If you want to send a
%token from the external code to the toolbox, the VDM-SL value {\tt
%  mk\_token(<HELLO>)} can e.g.\ be constructed in the following way:
%
%\begin{quote}
%\begin{verbatim}
%Record token(TOKEN, 1); 
%token.SetField(1, Quote("HELLO"));
%\end{verbatim}
%\end{quote}
%

\subsection{The \texttt{uselib} Path Environment}
\label{sec:uselibpath}

The name of the library is given with the {\tt uselib} option in the
specification. This location can be given in several ways:
\begin{itemize}
\item A full path name for the library (e.g.\ 
  /home/foo/libs/libmath.so)
\item Without path, but with the environment variable {\tt
    VDM\_DYNLIB} set. The library is searched for in every directory
  name in the {\tt VDM\_DYNLIB} environment variable. The environment
  variable could look like this: \path+/home/foo/libs:/usr/lib:.+ (Note
  that if you set the variable, you need a `.' if you want the search
  to look in the current directory too.)
\item Without path, and without the {\tt VDM\_DYNLIB} environment
  variable set. This means that the library is supposed to be located
  in the current directory.
\end{itemize}



\subsection{DL Module Initialisation}
\label{sec:init}

The first time the Toolbox command ``init'' is used all modules will
be loaded. The second time, the currently loaded modules are first
unloaded, then the modules are loaded again.

\subsubsection{Module Loading}
When a shared library file is loaded by the Toolbox, the following
will happen:  
\begin{itemize}
\item The global variables will be initialised. 
\item The toolbox executes the function InitDLModule(true) in each
  loaded module. 
\end{itemize}

\subsubsection{Module Unloading}
When a shared library file is unloaded by the Toolbox, the following
will happen:  
\begin{itemize}
\item The toolbox executes the function InitDLModule(false) in each
  loaded module. 
\item The global variables will be destructed.
\end{itemize}


\subsection{Creating a Shared Library}\label{sec:create}

The shared library must be compiled using the proper compiler as
required in Section \ref{sec:sysreq}.  In order to create an
executable shared library the code with the type conversion functions
must fulfill the following requirements:
\begin{itemize}
\item The type conversion functions must be enclosed in an {\tt extern "C"}
linkage specification, otherwise the functions will not be reachable from
outside the library.
\item The {\tt metaiv.h} header file, which is part of the {\tt VDM
    C++ Library}, must be included in the file with the type
  conversion functions as it contains prototypes of the type specific
  functions of the {\it VDM C++ Library}.
\end{itemize}

See the example makefiles in \ref{makefiles} for the actual compiler
flags to use.

On Unix a shared library should have the file extension {\tt ".so"} and 
a prefix {\tt "lib"}. (A version number for the shared library is not
required.) On Windows a shared library should have the file extension 
\texttt{".dll"}

\newpage
\bibliographystyle{iptes}
\bibliography{ifad}
\newpage
\appendix

\section{System Requirements}
\label{sec:sysreq}

To use the Dynamic Link feature the {\it VDM-SL Toolbox}
including a license to the Dynamic Link feature (a line with the {\tt
  vdmdl} feature must be present in the license file) and the {\it VDM
  C++ Library} are required. Furthermore a compiler is required in
order to create an executable shared library of the integrated code.

This feature runs on the following combinatins:
\begin{itemize}
\item Microsoft Windows 2000/XP/Vista and Microsoft Visual C++ 2005 SP1
\item Mac OS X 10.4, 10.5
\item Linux Kernel 2.4, 2.6 and GNU gcc 3, 4
\item Solaris 10
\end{itemize}

The compiler is required in order to create an executable shared
library of the integrated code. 
 
For installation of the Toolbox itself see Section 2 in \cite{UserMan-SCSK}
and for installing the {\it VDM C++ Library} see Section 2 in \cite{CGMan-SCSK}.



\section{Overview of the Trignometric Example}
\label{example}
In this example a simple user interface to trigonometric functions in
a specification is integrated in order to calculate the volume of a
circular cylinder. This example is available with the distribution in
the directory {\tt example}. To run the example do the following:
\begin{itemize}
  \item Edit the appropriate makefile, and set the paths at the top
    to where your libraries are.
  \item Compile the libraries
  \begin{description}
    \item[Linux] make -f Makefile.Linux
    \item[Solaris 10] make -f Makefile.solaris2.6
    \item[Microsoft Windows 2000/XP/Vista] make -f Makefile.win32
  \end{description}
  \item start the Toolbox, read the specifications, initialise and
    call the function. In the command line version of the Toolbox this
    would look as:
\begin{verbatim}
vdm> r cylio.vdm
Parsing "cylio.vdm" ... done
vdm> r cylinder.vdm
Parsing "cylinder.vdm" ... done
vdm> r math.vdm
Parsing "math.vdm" ... done
vdm> init
Initializing specification ...
vdm> p CYLINDER`CircCylOp()


 Input of Circular Cylinder Dimensions
  radius: 10
  height: 10
  slope [rad]: 1


 Volume of Circular Cylinder

Dimensions: 
   radius: 10
   height: 10
   slope: 1

 volume: 2643.56

(no return value)
\end{verbatim}
\end{itemize}

The {\it VDM-SL} document consists of a module {\sl CYLINDER} which
imports the DL modules {\sl MATHLIB} and {\sl CYLIO}. The DL module {\sl
  MATHLIB} provides an interface to the trigonemetric function $sinus$
and the value $\pi$. The module {\sl CYLIO} contains a function for
setting the dimension of the circular cylinder and for printing its
volume.  The module {\sl CYLINDER} also defines a record for
representing dimensions of a circular cylinder which is imported by
the DL modules.

For each DL modules there exists a corresponding shared library: {\sl
  MATHLIB} to {\tt libmath.so/math.dll} and {\sl CYLIO} to {\tt
  libcylio.so/cylio.dll}. 
 
The shared library {\tt libmath.so/math.dll} provides the trigonometric
functions $sine$, $cosine$ and the value $\pi$ which all are defined
in the C standard library. The code consists of the type conversion
functions.  The shared library \texttt{libcylio.so} (or
\texttt{cylio.dll} on Windows) provides
simple user 
interface functions: {\tt GetCircCyl} asks for the dimensions of the
cylinder and {\tt ShowCircCylVol} prints the cylinder dimensions and
its volume on the screen. The cylinder dimensions are represented by a
structure {\tt CircCyl} which has an equivalent definition in the
specification, the record type {\it CircCyl}.

\subsection{The Specifications}
This appendix contains the  module specifications.
\subsubsection{The Module \textsl{CYLINDER}}
\verbatiminput{trigno/cylinder.vdm}

\subsubsection{The Dynamic Link Modules \textsl{MATHLIB} and \textsl{CYLIO}}
\label{sec:math-cylio}

\verbatiminput{trigno/math.vdm}
\verbatiminput{trigno/cylio.vdm}


\subsection{The Shared Libraries}
This appendix contains the source files used to build the shared libraries.
\subsubsection{The MATHLIB shared library}
\verbatiminput{trigno/tcfmath.cc}

The type conversion function {\tt ExtSin} takes as argument an object 
of the class {\tt Sequence}, extracts the argument and converts it into an
object of the class {\tt Real}. This object is casted automatically in a
{\tt double} value by application of the $sin$ function.
The result value is converted into a Meta-IV value and returned to the
Toolbox.

\subsubsection{The CYLIO shared library}\label{sub:cylio}

This section contains the sources of the shared library {\tt
libcylio.so/cylio.dll}. 
The file {\tt cylio.cc} provides a simple interface for the input of the
dimensions of the circular cylinder and the printout of the volume.

The file {\tt tcfcylio.cc} contains the type conversion functions.
The type conversion functions also show the transformation of the
circular cylinder values between the different representations of
the type ``CircCyl'' used in C++ and in the Toolbox.

\verbatiminput{trigno/cylio.h}
\verbatiminput{trigno/cylio.cc}
\verbatiminput{trigno/tcfcylio.cc}
  
\subsection{Makefiles}\label{makefiles}


\subsubsection{Linux}\label{makefiles:linux}

Here is an example makefile for Linux. 

\verbatiminput{trigno/Makefile.Linux}


\subsubsection{Solaris 2.6}\label{makefiles:sunos5}
Here is the example makefile for Solaris 10

\verbatiminput{trigno/Makefile.solaris2.6}


\subsubsection{Windows}\label{makefiles:win}
Here is the example makefile for Microsoft Windows, using GNU make.

\verbatiminput{trigno/Makefile.win32}




\section{Trouble Shooting}
Once you have started creating your own type conversion functions, you
will discover that suddenly strange error messages from the VDM C++
library may occur.  This appendix will help you providing an idea of
how to debug your libraries.

The best thing is to start {\tt gdb} as usually, and debug through the
code, to see where it goes wrong. This cannot be done straight away,
since the symbols defined in the libraries are not known, when {\tt
  gdb} is started.

The solution is first to debug the specification with {\tt vdmde}, and
when it has been discovered which external function goes wrong, a {\tt
  main()} which calls that particular function can be created and
linked together with your libraries. The main procedure could look
like this:

\verbatiminput{trigno/main.cc}

\subsection{Segmentation Fault}

Another problem you may have is that the toolbox crashes with a
segmentation fault. This might very well be a problem in your external
library. Again, {\tt vdmde} is started in order to see in which
function the problems occur.

There are three obvious things which can go wrong in the interface
between the Toolbox and your code resulting in such an error:

\begin{enumerate}
\item Your functions do not have the correct signature:\\
  {\tt void} {\em function\/}{\tt (Sequence }{\em args\/}{\tt )}\\
  or {\tt Generic} {\em function\/}{\tt (Sequence }{\em args\/}{\tt )}
\item A function which should return {\tt Generic} does not return any
  value.
\item The external code contain global uninitialised values or objects
  (see Section \ref{sec:globalvalues}).
\end{enumerate}

\subsection{Using Tcl/Tk}

If you wish to use Tcl/Tk in a dynamically linked library together
with the Toolbox you should look at the other examples which have been
made for this combination \footnote{There is no exercise being offered now.}.
It is important that you do not call {\tt Tk\_MainLoop} 
if you are using {\tt vdmgde} (the graphical version of the Toolbox) 
because {\tt vdmgde} itself is implemented with Tcl/Tk and 
therefore it will enter a deadlock situation where Tcl/Tk will be waiting 
for the main Toolbox window to close in case {\tt Tk\_MainLoop} is called.

One solution (which is used in the other examples) is to create a loop
which calls {\tt Tk\_DoOneEvent} and terminates when the Tcl code has
set a Tcl variable {\tt Done} non zero. Initially, the Tcl code sets
{\tt Done} to zero and when the dialog window is closed {\tt Done} is
set to 1.

\subsection{Global Objects and initialisation of Global Values}
\label{sec:globalvalues}

On all supported platform (Windows, Mac, Linux, Solaris) C++ object
constructors are called when the shared object is compiled and linked
as described above.

\subsection{Standard input and output under Microsoft Windows}

Since the graphical version of the Toolbox runs as a GUI application
under Windows, standard input and output streams are not
available. Therefore shared libraries that attempt to use
\texttt{cin}, \texttt{cout} and \texttt{cerr} will not work as
expected. 

\end{document}

