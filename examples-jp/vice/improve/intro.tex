\section{Introduction}

This memo intends to provide a series of proposals for enhancing VDMTools
in its future innovative development. Because of the main emphasis on
real-time embedded and distributed systems development from the CSK
Corporation, the suggestions made in this memo focus on exactly that:
enhancing VDMTools for development of such systems. The intent is that
CSK Corporation can make use of the body of knowledge from a collection
of VDM tool support experts (mainly placed throughout Europe, both
industrial and academic users of VDMTools) and use them as external
consultants and/or subcontractors. One can imagine this being arranged
under either time-and-material contractual terms or on firm-fixed-price
terms for one or more of the improvements. \\

Historically, VDM and all the other comparable model-oriented formal
specification approaches (e.g. RAISE, Z and B) have primarily been used
for the development of sequential computer based systems. The majority of
formal approaches that have been used for the development of concurrent
computer systems use explicit focus on the channels used for communication
between the processes (e.g. CCS and CSP). A number of the formal approaches
have been extended in different ways to include a notion of time (e.g. Timed
CSP and the VICE VDM++ version). However, virtually none of the existing
formal approaches are able to deal with combination of concurrency, time
and distributed architecture in a way that would be appropriate for the
development of real-time embedded and distributed systems. This memo
includes suggestions for improving the VICE VDM++ technology with
capabilities enabling exactly this combination and we strongly believe
that if these suggestions are incorporated into VDMTools a really powerful
capability enabling better development of realistic embedded systems will
become available. In particular it will become easy for a specifier to
experiment with different distributed architectures at a very early stage
in the design and thus validate important properties in a cost-efficient
fashion. \\

The VICE (VDM++ In a Constrained Environment) project was supported by
the European Union in collaboration with Matra BAe Dynamics and it was 
carried out shortly before Peter Gorm Larsen left IFAD. This project aimed at
improving the VDMTools capabilities in particular in the real-time embedded
area and a lot of progress was made, but it was never properly commercialised
because of Peter Gorm Larsen�s departure from IFAD. The application used for
a case study in this project was a missile guidance system with a single
processor. Thus, in this project there was not any focus on distribution
between different processors. Nowadays there is however significant more
emphasis on the distributed aspects of real-time embedded systems. The
suggestions made in this memo are thus a continuation of the efforts made
in the VICE project. \\

One of the triggers for this memo is the paper \cite{viceeval}
written by Marcel Verhoef of Chess Information Technology (NL) that was
presented at the first Overture workshop\footnote{This paper is available
at \textit{http://www.cs.ru.nl/research/reports/info/ICIS-R05033.html}.},
entitled \textit{``On The Use of VDM++ for Specifying Real-Time Systems''}.
Marcel is a long-time VDM enthusiast; he made the first version of the
type-checker for VDMTools when he was still a student at the Technical
University Delft (NL) doing his MSc thesis project at IFAD in Denmark
back in 1992. Later, he used VDMTools on several large scale industrial
projects and he is also co-author of the recently published VDM++ book
\cite{vdmbook}. Currently he is part of the BODERC research project at
the Embedded Systems Institute (see \textit{http://www.esi.nl}) where
he works on his PhD thesis. His PhD supervisor is professor Frits Vaandrager
at the Radboud University Nijmegen (NL).\\

In \cite{viceeval}, Marcel took the 6.7.27 version of the VICE tool
and used it to model a large distributed industrial real-time embedded
system (a digital control system for a high-volume office printer). The
paper lists many areas where both the tool as well as the language could
be improved to better fit this class of systems -- in his belief many
changes are mandatory in order to be sufficiently productive and successful
in a commercial environment. The discussion at the conference let to a
joint research session with Peter Gorm Larsen held at Aarhus in October
2005. We reviewed the problems identified and discussed the suggested
solutions. This fruitful session actually let to several more insights
that could solve problems that are not even mentioned in the Overture
paper. \\

Currently, we are targeting a scientific paper for FM'06 that describes
these (language) improvements. Basically, they can be categorised as
follows (in random order):

\begin{itemize}
\item Enhancements to the visualisation capabilities of VDMTools;
\item Enhancements to the static and dynamic analysis capabilities of VDMTools;
\item Enhancements to the VDM++ notation and the corresponding tool support;
\item Improvements to the UML coupling.
\end{itemize}

Prior to presenting the different categories this introduction is followed
by listing the tasks that needs to be carried out to in order to demonstrate
the proof of concept of the improve suggestions made in this memo. Afterwards, 
each improvement category will be detailed further below.  

Finally, this memo is completed by a case study, describing a typical
real-time embedded and distributed system. After introducing the case,
we first show how a suitable VDM++ model can be produced with the
existing VICE VDM++ technology. Then we shown how the VDM++ model
could potentially be formulated and validated with the suggested
improvements. At the end of the VDM++ model using the existing VICE technology
it is also illustrated why it is hard with the currently available
capabilities to discover where potential time requirements are violated
and why this is the case.
We believe that it is very likely that any developments
made with VDMTools for embedded systems has a substantial risks for
failure if the suggested improvements are not realised. Hopefully the
new approach will be able to make it easier to validate such requirements
and when violations are discovered faster to correct the model.

