\subsection{Improvement of the VDM++ language}

The VICE notation was recently evaluated by Marcel in his paper
\cite{viceeval} presented at the first Overture workshop explaining the
problems when describing distributed embedded real-time systems with VICE.
The main challenges he identified with the current notation are:

\begin{itemize}
\item The VDM++ notation is by nature synchronous; operation calls are
either blocked on an permission predicate or executed in the context of
the thread of control of the caller (an active object with its own
\verb+thread+ clause in the class definition). This is very cumbersome
when describing embedded systems, which are typically reactive (and 
therefore asynchronous) by nature. Of course, an event loop can be
specified to simulate the asynchronicity, but it would clobber the
specification greatly and lowers the productivity of the user. The
specification is intrinsically bigger than necessary and the validation
of the model is also more complex because of the increased model size.
Users are typically overwhelmed by the amount of run-time data that such
a ``hand-coded'' event loop generates. The language should therefore
allow for \textit{asynchronous} operation calls to overcome this problem. 

\item Similarly, the active object is currently the ``owner'' of the thread
of control. This is ok when all threads are executed on the same processor
(this is the current implementation in VICE) but it really complicates
matters when distributed systems are modelled. Note that when describing
embedded systems, this necessity is paramount, because the environment of
the system can always be viewed as the ``second processor'' that works in
full-parallel to and independent from the embedded system; both have their
own behaviour and (timing) requirements that are only influenced by the
(timed) exchange of stimuli and responses. Therefore, the ownership of the
thread of control should move from the ``active class'' to the (physical
or virtual) processor on which the class instance is deployed. The behaviour
of the system is then determined by the scheduling policy on that processor
and its total capacity (available computing power).  By doing this, it would
be possible to describe deployment of software over a distributed computer
system, also taking into account that one processor might be slower than
another, leading to different response times. This would be a great step
forward; as far as we know this has not been done before and it has a
dramatic practical impact for describing industrial applications, where
most systems nowadays are in fact already multi-processor solutions. 

\item Last but not least, we have some ideas that would also allow
describing the interconnection between the processors (the network) at
a very high-level of abstraction. Inter task communication is now considered
instantaneous in VDM++ (or must be encoded explicitly using duration
statements) which is far from reality. The network deployment view we
envisage gives rise to include communication overhead into the simulation
model at relative little extra cost (for both describing and analysing the
system). In combination with the asynchronous operation calls, all typical
styles of inter processor (and inter task) communication can be described:
synchronous, asynchronous and publish/subscribe, while remaining at a
high-level of abstraction when writing the specification.

\end{itemize}

We believe that these changes can be achieved with limited impact to the
existing syntax and semantics of the language and the tool. We are currently
working on a scientific paper that describes these conceptual changes to the
VDM++ language. This should be seen as a ``proof of concept'' description
that could be used as a guideline to the full-blown implementation in
VDMTools. The first draft of the paper is expected to be ready in
February 2006. A case study illustrating the effect of these suggested
improvements is presented in Sections~\ref{sec:viceorg} and \ref{sec:vicenew}.
However, we do emphasise that this is ``work in progress''. For example,
the specification using the improved language has not been mechanically
verified because tool support is simply lacking.

