\subsection{Proof of Concept of Enhanced VDM++ Real-Time Features}

In order to assess whether the approach advocated in this memo for 
improving the VDMTools technology in the development of embedded real-time 
and distributed systems, this subsection describes the tasks suggested to 
be carried out to form a proof of concept. These tasks must produce:

\begin{enumerate}
\item An update of the VICE dynamic semantics specification reflecting 
      the suggested 
      semantic modifications in the improved real-time system;
\item A demonstration using the car radio-navigation example shown 
      in this memo about the suggested improvements;
\item An updated scanner and parser that can produce the updated 
      abstract syntax to the dynamic semantics specification;
\item An updated version of ShowVICE that will be able to demonstrate 
      the car radio-navigation example and demonstrate the analysis desirable; 
\item A high level product level functionality description that can be used
      to promote the use of VDMTools for the development of embedded
      real-time and distributed systems using the newly suggested approach; and
\item A guidelines document that at a more technical level is able to 
      provide guidelines as to how the new sugegsted approach with advantage
      can be used for the development of embedded real-time and distributed 
      systems. This document will take the existing guidelines document from 
      the VICE project as a starting point.
\end{enumerate}

It is recommended that the first 5 of these delivarables are produced by 
Peter Gorm Larsen and Marcel Verhoef jointly, whereas the last deliverable 
just as well could be made by Professor Araki's group (and in this way 
probably faster be translated into Japanese).