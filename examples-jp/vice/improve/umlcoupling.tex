\subsection{Enhancement of the UML coupling for VDMTools}

The ``Rose link'' was originally made to couple VDMTools with Rational Rose.
Meanwhile, UML has developed substantially and is now an updated standard -
version 2.0. It would be very advantageous for all users of VDMTools that
wish to use VDM++ in conjunction with class diagrams from UML if this new
standard would be adopted. The enhancements could for example include:

\begin{itemize}
\item Rather than being bound to Rational Rose only, the UML link could be
implemented using XMI (the UML model interchange format) such that all UML
tools, that support this exchange standard for UML models, could be used
(including public domain tools). This would remove the current ``vendor
lock-in'' to Rational Rose for VDMTools. This is in particular important
for the real-time and embedded market, where I-Logix Rhapsody seems to
have a much larger market penetration than Rational Rose. 

\item The I-Logix company have produced a system engineering process called
``Harmony'' that seem very appropriate for the development of embedded
real-time systems. This method and the recently released SysML (the Systems
Engineering modelling language, also from the OMG in collaboration with the
International Council of Systems Engineers INCOSE) should be analysed further
for better integration with VDMTools for the early phases of development. 

\item As a minor point, support for packages and hierarchical classes
in UML 2.0 could be taken into account in VDMTools. Similarly, state
transition diagrams from UML could easily be imported into a VDM++ model,
provided that asynchronous operation calls are supported in the language.
\end{itemize}

The round-trip engineering approach advocated by the existing VDMTools
is still very worthwhile in the context of UML\,2.0. The new UML notation
still has no agreed syntax (and semantics) for the specification of
(for example) operations -- which is a natural fit with VDM++, as
described for example in the VDM++ book \cite{vdmbook}. Do note however
that Stephen Mellor is leading an OMG activity currently to define a (new)
language for the creation of executable UML models, something that VDM++
already provides but the majority of UML users is probably not aware of.
To prevent OMG re-inventing the wheel once more (as they did with OCL),
it is suggested that CSK, in collaboration with their academic partners,
makes a case for acceptance of VDM++ as a suitable and already existing
and industry proven solution for the specification of executable UML
models. This should be actively pursued and pushed as soon as possible. \\

OMG also promotes the Model Driven Architecture approach to systems
analysis and design. While the coupling to UML already implements parts
of that strategy, the current VDMTools implementation is not well-suited
for this. It was just not designed with MDA in mind. The Overture open
source initiative (\textit{http://www.overturetool.org}) attempts to
solve part of this problem by providing access to a standardized XML
format for the abstract syntax of VDM specifications. We suggest that
the Overture effort is actively supported from within CSK. Overture is
the ideal platform to get academic researchers involved in the
development of new tools and ideas as proposed in this note. It can
become a breading ground for new ideas and an excellent opportunity
for community building. \\

The above two paragraphs indicate that the current work on UML\,2.0
and MDA is pretty much orthogonal to the improvements to VICE proposed
in this note. These issues are not hampering each other but the
relationship between VDM++, UML and MDA can certainly be strengthened
by implementing the suggested language and tool improvements. In our
opinion, it will also leverage the market potential for VDMTools.

