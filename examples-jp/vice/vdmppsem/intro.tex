\section{Introduction}

\noindent \tbw.

\begin{enumerate}
\item Introducing VDM++ (move from start of Section~2?).
\item Explain the typical properties of distributed embedded
software development, use \cite{Lee05}. Typically SFM are used but
they are limited when complex algorithms come into play (for example
higher levels of control and error handling). Also introduce commonly
accepted RTE notions such as jitter, clock drift.
\item Aim of modeling is to increase the insight into the system
under construction at minimal cost.
\item Make minimal changes to both VDM++ syntax, static and
dynamic semantics. Keep the application specification as much
as possible free from (hardware) architecture specific knowledge.
\item Concrete ASCII syntax for case study in VDM++ and mathematical
syntax for semantics in VDM-SL.
\item Make clear to the reader the distinction between models at
the VDM++ level (application models?) and at the meta-level
(semantic models?). Align all sections accordingly.
\end{enumerate}

\subsection{Contribution of this paper}

\noindent \tbw.

\begin{enumerate}
\item Enabling (bridging the gap between) formal software engineering
and hardware / networking disciplines.
\item Properly deal with the notion of deployment on the 
computation and communication level.
\item Improvement of VDM++ technology to better support distributed
and embedded real-time systems. Ability to describe a whole new class
of systems.
\end{enumerate}

\subsection{Related work}

\noindent \tbw.

\begin{enumerate}
\item Certainly we have to reflect on TrueTime, Ptolemy and Giotto
and show why we are different.
\item Check the dan.bib file for more related work, just like
\cite{Huijsman&93} and \cite{Garlan&90b} -- product families.
\item Discussion notion of deployment in UML and AADL.
\end{enumerate}
