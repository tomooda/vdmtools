\section{The VDM++ approach to real-time}
\label{sec:vdmpp}

The formal notation VDM++ is an object-oriented, model-based specification
language, and is largely a superset of the ISO standardized notation VDM-SL
\cite{ISOVDM95}. VDM++ was originally developed in the ESPRIT project
\textsc{AFRODITE} and was subsequently improved by the Danish company
IFAD. The real-time extensions to VDM++ where developed as part of the
ESPRIT project \textsc{VICE}: ``VDM++ In a Constrained Environment''.
Both the VDM++ and VICE notations are supported by the industry strength
\vdmtools, which is now owned and maintained by the Japanese
company CSK \cite{CskVdmHome}.

VDM++ provides a precise and unambiguous notation for
analysis of requirements and allows for early validation through testing
and debugging using \vdmtools. In this way it is possible to bring
testing activities forward to the specification phase of the
development life-cycle. This pragmatic approach to applying formal
techniques for systems development has been instrumental for a number
of very successful applications in industry, for example \cite{Berg&99a},
\cite{Horl&00b} and Felicia Networks. In Section~\ref{subsec:vdmppold}
we present a short overview of the VDM++ language and its real-time
extensions, both in terms of capabilities and limitations.
In Section~\ref{subsec:vdmppnew} we will investigate how to overcome
the limitations by introducing some extensions to the notation and its 
associated tool support.

\subsection{The existing VDM++ capabilities and limitations}
\label{subsec:vdmppold}

In VDM++, a complete formal specification consists of a collection of 
class specifications. A class specification has the following components:

\begin{description}
\item[Class header:] This contains the class name declaration and
inheritance information (single or multiple).

\item[Instance variables:] The state of an object consists of
variables which can be of simple types, types
such as sets, sequences, tuples, records and maps, and object
references (the clientship relation). Instance variables can have
invariant and initial expressions.

\item[Operations:] Class methods that may be defined implicitly,
explicitly (through imperative statements),
or as a mixture of both. The implicit style uses pre and
post condition expressions. 

%\item[Functions:] Functions are similar to operations except that the
%body of a function is an expression rather than a statement. Also,
%functions are not allowed to refer to instance variables.

\item[Synchronization:] Operation invocation is defined with 
synchronuous semantics (rendez-vous). It is possible to specify the
circumstances in which an operation may be executed using a
\textit{permission predicate} for the operation. This predicate is
over the instance variables of the object, and also \textit{history
variables} for that object. A history variable can be used to count
the number of requests, activations and completions for an operation
on that object.

\item[Thread:] In VDM++ active objects are considered to
model active world entities. An object can be
made active by the specification of a thread. A thread is a sequence
of statements which are executed to completion, at which point the
thread dies. 
\end{description}

\vdmtools\ is a comprehensive suite of tools for
the analysis and validation of formal models described in
VDM++. Different variants of \vdmtools\ exists but in this article we
limit our efforts to the version resulted from the VICE project and an
auxiliary tool called \texttt{ShowVICE} \cite{Verhoef05}. The only part of 
\vdmtools\ important for this article is the interpreter. It is able 
to perform execution of VDM++ models. It has functionality enabling 
debugging using breakpoints and single/multiple stepping directly at the
VDM++ level. Execution of thread-based models, with different scheduling 
policies selected by the user e.g.\ cooperative round-robin scheduling.

Specifically in the VICE version of \vdmtools\ the notion of time and
periodic threads is available. 
In general the approach inside the interpreter is that the time of the
previous instruction is recorded in an internal variable. The task
switching overhead is a constant defined by the user. The
execution time for statements executed since the previous event, is
the sum of the execution times for each such statement. 
However, there is a notion of a duration statement.
A duration is an estimate of how much time a particular portion of a
VDM model will take to execute, in the implementation, on the target
processor. The information provided by a duration statement is
used to override the default execution time calculated for that
portion. 

The basic idea of the approach is to simulate the timing behaviour of
the target processor within the interpreter. To achieve
this the interpreter maintains an internal variable which
corresponds to the clock of the target processor i.e.\ the clock of the
target processor will be simulated. The interpreter will adopt the
same scheduling algorithm as that intended for the final system. During
execution of the model a number of events will occur: 
\begin{itemize}
\item Swapping in and out of threads
\item Operation requests, activations and completions
\end{itemize}
We call such events, trace events 
\footnote{\textsc{TR:}For the purposes of this paper we
restrict our interest to the swapping in and out of threads.}.

Each trace event is logged in a trace file, with the time at which the
event occurred. This time is the reading of the clock on the target
processor as recorded by the interpreter when the event occurred. To
maintain the internal variable representing the target processor's
clock, selected portions of the VDM++ model are enhanced with duration
information, a file of default duration information is utilized and
the user provides a default task switching overhead. After the
completing of an execution it is then possible to make use of the
\texttt{ShowVICE} tool to automatically get information about the trace
produced in the form of an extended sequence diagram with time annotations
and thread context.

\subsubsection{Limitations}

\begin{enumerate}
\item VDM++ is in fact a synchronous language, there is no notion of events
\item VDMTools only supports uni-processor multi-threading
\item \textsc{TR:} No access to the ``global time'' (if space available also
new permission predicates).
\item No notion of distribution in the language. Describe 5 basic communication
patterns that we want to support.
\item Fixed format for trace file.
\item Only absolute time progress (\texttt{duration}) is supported, no relative
notion of time (depending on CPU context or BUS properties).
\end{enumerate}

\subsection{Suggestions to overcome the limitations of VDM++}
\label{subsec:vdmppnew}

\noindent To Be Written.

\begin{enumerate}
\item Introduction of \texttt{async} operation calls.
\item Introduction of \texttt{system}, \texttt{CPU}, \texttt{BUS}.
\item Introduction of \texttt{cycles}.
\item Flexible trace file.
\end{enumerate}
