%
% $Id: radnavnew.tex,v 1.1 2005/11/21 05:04:09 vdmtools Exp $
%

\section{The new stuff}

\begin{alltt}
system RadNavSys

instance variables
  -- create an MMI class instance
  static public mmi : MMI := new MMI();
  -- define the first CPU with fixed priority scheduling and 22E6 MIPS performance
  CPU1 : CPU := new CPU (CPU`FP, 22E6);

  -- create an Radio class instance
  static public radio : Radio := new Radio();
  -- define the second CPU with fixed priority scheduling and 11E6 MIPS performance
  CPU2 : CPU := new CPU (CPU`FP, 11E6);

  -- create an Navigation class instance
  static public navigation : Navigation := new Navigation();
  -- define the third CPU with fixed priority scheduling and 113 MIPS performance
  CPU3 : CPU := new CPU (CPU`FP, 113E6); 

  -- create a communication bus that links the three CPU's together
  BUS1 : BUS := new BUS (BUS`CSMACD, 72E3, {CPU1, CPU2, CPU3})

operations
  public RadNavSys: () ==> ()
  RadNavSys ()
    ( -- deploy mmi on CPU1
      CPU1.deploy(mmi);
      CPU1.setPriority(HandleKeyPress,100);
      CPU1.setPriority(UpdateScreen,90);
      -- deploy radio on CPU2
      CPU2.deploy(radio);
      CPU2.setPriority(AdjustVolume,100);
      CPU2.setPriority(DecodeTMC,90);
      -- deploy navigation on CPU3
      CPU3.deploy(navigation);
      CPU3.setPriority(DatabaseLookup, 100);
      CPU3.setPriority(DecodeTMC, 90)
      -- starting the CPUs and BUS is implicit )

  static public wait: () ==> ()
  wait () == skip;

  sync
    per wait => mmi.cnt > 30

end RadNavSys
\end{alltt}

\begin{alltt}
class MMI

instance variables
  public cnt : nat := 0

operations
  async public HandleKeyPress: nat ==> ()
  HandleKeyPress (pn) ==
    ( duration (1E5) cnt := cnt + 1;
      cases (pn):
        1 -> RadNavSys`radio.AdjustVolume(),
        2 -> RadNavSys`navigation.DatabaseLookup()
      end );

  async public UpdateScreen: nat ==> ()
  UpdateScreen (-) ==
    duration (5E5) skip;

end MMI
\end{alltt}

\begin{alltt}
class Radio

operations
  async public AdjustVolume: () ==> ()
  AdjustVolume () ==
    ( duration (1E5) skip;
      RadNavSys`mmi.UpdateScreen(1) );

  async public HandleTMC: () ==> ()
  HandleTMC () ==
    ( duration (1E6) skip;
      RadNavSys`navigation.DecodeTMC() )

end Radio
\end{alltt}

\begin{alltt}
class Navigation

operations
  async public DatabaseLookup: () ==> ()
  DatabaseLookup () ==
    ( duration (5E6) skip;
      RadNavSys`mmi.UpdateScreen(2) )

  async public DecodeTMC: () ==> ()
  DecodeTMC () ==
    ( duration (5E6) skip;
      RadNavSys`mmi.UpdateScreen(3) )

end Navigation
\end{alltt}

\begin{alltt}
class VolumeKnob

thread
  periodic (p,j,d)
    duration (0)
      while (true)
        RadNavSys`mmi.HandleKeyPress(1)

end VolumeKnob
\end{alltt}

\begin{alltt}
class InsertAddress

thread
  periodic (p,j,d)
    duration (0)
      while (true)
        RadNavSys`mmi.HandleKeyPress(2)

end InsertAddress
\end{alltt}

\begin{alltt}
class TransmitTMC

thread
  periodic (p,j,d)
    duration (0)
      while (true)
        RadNavSys`radio.HandleTMC()

end TransmitTMC
\end{alltt}

\begin{alltt}
class World

operations
  public RunScenario1 : () ==> ()
  RunScenario1 () ==
    ( start(new RadNavSys());
      startlist({new VolumeKnob(), new TransmitTMC()});
      RadNavSys`wait() );

  public RunScenario1 : () ==> ()
  RunScenario1 () ==
    ( start(new RadNavSys());
      startlist({new InsertAddress(), new TransmitTMC()});
      RadNavSys`wait() );

end World
\end{alltt}

\begin{alltt}
new World().RunScenario1()
\end{alltt}
