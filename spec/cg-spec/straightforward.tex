% WHAT
%    The main file of the documentation of the specification of 
%    the code generator.
% FILE
%    $Source: /home/vdmtools/cvsroot/toolbox/spec/cg-spec/straightforward.tex,v $
% VERSION
%    $Revision: 1.24 $
% DATE
%    $Date: 2001/02/21 13:38:48 $
% FORMAT
%    $State: Exp $
% PROJECT
%    Afrodite - ESPRIT III programme, project no. 6500.
% STATUS
%    Under development.
% AUTHOR
%    $Author: paulm $
% COPYRIGHT
%    (C) 1993 IFAD, Denmark

\documentclass[a4paper,dvips]{article}
\usepackage[dvips]{color}
\usepackage{vdmsl-2e}
\usepackage{longtable}
\usepackage{alltt}
\usepackage{makeidx}
\usepackage{ifad}

\definecolor{covered}{rgb}{0,0,0}      %black
%\definecolor{not_covered}{gray}{0.5}   %gray for previewing
\definecolor{not_covered}{gray}{0.6}   %gray for printing
% \definecolor{not_covered}{rgb}{1,0,0}  %read

%\documentstyle[vdmsl_v1.1.31,alltt,makeidx,ifad,a4]{article}

#ifdef VDMSL
\newcommand{\VDM}{VDM-SL}
#endif //VDMSL
#ifdef VDMPP
\newcommand{\VDM}{VDM++}
#endif //VDMPP

\newcommand{\StateDef}[1]{{\bf #1}}
\newcommand{\TypeDef}[1]{{\bf #1}}
\newcommand{\TypeOcc}[1]{{\it #1}}
\newcommand{\FuncDef}[1]{{\bf #1}}
\newcommand{\FuncOcc}[1]{#1}
\newcommand{\ModDef}[1]{{\tiny #1}}


\newcommand{\MCL}{Meta-IV Class Library}
\newcommand{\TBW}{TO BE WRITTEN}
\newcommand{\NYI}{Notice, that this function/operation is not fully
  specified in order to capture a possible integration with another
  library than \MCL{}.}
\newcommand{\nfs}{{\em not fully specified\/}}

\makeindex

\begin{document}

\special{!userdict begin /bop-hook{gsave 220 30 translate
0 rotate /Times-Roman findfont 21 scalefont setfont
0 0 moveto (CONFIDENTIAL) show grestore}def end}

#ifdef VDMSL
\docdef{Specification of the VDM-SL Code Generator}
{ The IFAD VDM Tool Group \\
  The Institute of Applied Computer Science}
{\today,}
{IFAD-VDM-12}
{Report}
{Released}
{Confidential}
{}
{\copyright IFAD}
{\item[V1.0] First version.
\item [V1.1] Reducing the size and improving the readability
  of the generated code mainly by reducing the number of
  temporary variables. 
\item [V1.2] A number bugs have been fixed. Dead code has been removed
  and more test cases have been added to the test environment in order
  to obtain full coverage.
\item[V1.3] Major changes, in particular to the CPP module where
context information has been added (to be ignorred by the code generator). } 
{}
#endif // VDMSL

#ifdef VDMPP
\docdef{Specification of the VDM++ Code Generator}
         {IFAD VDM Tool Group \\
          The Institute of Applied Computer Science}
         {\today,}
         {IFAD-VDM-29}
         {Report}
         {Released}
         {Confidential}
         {}
         {\copyright IFAD}
         {\item[V1.0] First version.
	 \item[V1.3] Major changes, in particular to the CPP module where
context information has been added (to be ignorred by the code generator).}
         {}
#endif // VDMPP
\renewcommand{\thepage}{\roman{page}}

\tableofcontents
\newpage
\renewcommand{\thepage}{\arabic{page}}
\setcounter{page}{1}

\parskip12pt
\parindent0pt

#ifdef VDMSL
\section{Introduction}

This document contains the specification of the VDM-SL code generator.
In this version of the code generator the following constructs are
not supported:

\begin{itemize}
\item Expressions:

  \begin{itemize}
  \item Lambda.
  \item Compose, iterate and equality for functions.
  \item Function type instantiation expression. However, the code
    generator supports function type instantiation expression in
    combination with apply expression, as in the following example:

\begin{quote}
\begin{verbatim}
Test:() -> set of int
Test() ==
  ElemToSet[int](-1);

ElemToSet[@elem]: @elem +> set of @elem
ElemToSet(e) ==
  {e}
\end{verbatim}
\end{quote}

  \end{itemize}

\item Statements: 

  \begin{itemize}
  \item `{\sf using}' in call statement.
  \item AlwaysStmt.
  \item ExitStmt.
  \item TrapStmt.
  \item RecTrapStmt.
  \end{itemize}

\item State designator:

  \begin{itemize}
  \item Field reference
  \item Map reference
  \item Sequence reference
  \end{itemize}

\item Type bind in:

  \begin{itemize}
  \item Let-be-st expression/statements.
  \item Sequence, set and map comprehension expressions.
  \item Iota and quantified expressions.
  \end{itemize}

\item Patterns:

  \begin{itemize}
  \item Set union pattern.
  \item Sequence concatenation pattern.
  \end{itemize}

\item Local Function Definitions.

\item Higher order function definitions.

\item Function Values. 

\end{itemize}

However, the code generator are able to generate compilable code for
specifications including the former constructs. But the execution of
the code will result in a run-time error, if a branch containing a
unsupported construct is executed. Consider the following function
definition: 

\begin{quote}
\begin{verbatim}
f: nat -> nat
f(x) ==
  if x <> 2 then
    x
  else
    iota x : nat & x ** 2 = 4
\end{verbatim}
\end{quote}

In this case {\tt f} can be code generated and compiled.  The compiled
C++ code corresponding to {\tt f} will result in run-time error, if
{\tt f} is applied with the value 2.

Note, that the implementation of the code generator will give a
warning whenever a unsupported construct is ``code generated''.


\subsection{Semantical Differences Between the Generated Code, the
  Interpreter, and ISO VDM-SL}

\begin{itemize}
\item Value Definitions: In order to execute the specification, it is
  required by the interpreter and the code generator, that values must
  be defined before they are used, in the order they are defined in a
  module.

\item Inconsistence between the scope rules used by the interpreter,
  the code generator and ISO VDM-SL in let/define
  expressions/statements. Consider the following expression:

  \begin{verbatim}
let a = 1 in
  let a = 2,
      b = a in
    a + b
  \end{verbatim}

  The interpreter will evaluate this expression to 3, whereas the
  generated code will evaluate the expression to 4. In the
  generated code local values are ``visible'' at the level where they
  are defined, in the order they are defined. In the interpreter local
  values are not in the scope at the level where they defined. In ISO
  VDM-SL local values may be used in any order at the level there they
  are defined.

\item Iota expression and unique existential expression: In the
  interpreter and the generated code a true valued unique existential
  expression implies that the corresponding iota expression can be
  computed successfully. Consider the following expression:

  \begin{verbatim}
exists1 {a,b} in set {{1,2}} & a > b 
  \end{verbatim}

  In this case both {\sf true} and {\sf false} are legal solutions.
  However, the corresponding iota expression has only a solution in
  the case where {\tt a} $=2$ and {\tt b} $=1$. Thus, in the
  interpreter and the generated code a unique existential expression
  must evaluate to {\sf true} in order to find a solution to the
  corresponding iota expression.

\end{itemize}


\section{Overall Structure}

This section describes the {\bf design} of the specification, the {\bf
strategy} used in the code generator, and the specification {\bf style}.

\subsection{Design}

The specification is divided into several modules:

\begin{description} 

\item[MOD:] (MOD is an abbreviation of MODule ). This module describes the generation of code of
modules (Not fully specified).

\item[CG:] (CG is an abbreviation of Code Generator). This module is
  to be considered as the main module, it generates code corresponding
  a whole specification.


\item[FD: ] (FD is an abbreviation of Function Definitions ).
  Providing functions generating code for function and operation
  definitions.

\item[EXPR: ] (EXPR is an abbreviation of EXPRession). Providing
  functions generating code corresponding to expressions.

\item[PM: ] (PM is an abbreviation of Pattern Matching). Providing
  functions generating code of pattern match. 
  
\item[STMT: ] (STMT is an abbreviation of STateMenT). Providing
  functions generating code corresponding to statements.

\item[TD: ] (TD is an abbreviation of Type Definitions). Providing
  functions generating code for type definitions.

\item[SD: ] (SD is an abbreviation of State Definitions). Providing
  functions generating code for state definitions.

\item[VD: ] (VD is an abbreviation of Value Definitions). Providing
  functions generating code for value definitions.

\item[CPP:] (CPP is an abbreviation of C Plus Plus). Describing the
  abstract syntax of C++.

\item[BC: ] (BC is an abbreviation of Building C++). The construction
  of part of an abstract tree of C++ is described in this module. The
  main idea behind this module is divide the specification of the code
  generator into parts of the same nature. This module also provides
  functions for naming temporary variables and auxiliary functions.

\item[AS: ] (AS is an abbreviation of Abstract Syntax). Describing the
  abstract syntax of the VDM-SL/VDM++.

\item[DS: ] (DS is an abbreviation of Data Structures). Providing
  functions generating code corresponding to functions on VDM data
  structures, for example providing a function which generates code
  corresponding to a {\bf union} operation between two sets. That is,
  the basic data type implementation is described in this module.

\item[AUX:] (AUX is an abbreviation of AUXiliary). Provides auxiliary
  functions and operations to the rest of the specification.

\item[REP: ] Type Representations Module. Provides the type
  definitions for the semantic domains which are used internally in
  the static semantics to model the types from the abstract syntax and
  the derived types.


\end{description}


\subsection{Strategy}

\subsubsection{Scopes}

In the generated code we use C++ scope to model VDM scope.  The idea
is to use variables which are automatically created and destroyed
inside blocks in C++. Whenever code is to be generated of a subtree of
for example an expression it is assumed that variables appearing in
that subexpression, which are not declared in that expression, are
declared in an outer block. An example:

\begin{quote}
\begin{verbatim}
fct: () -> nat
fct() ==
let a=2
in let a=3,
       b=4
   in a+b
\end{verbatim}
\end{quote}

Using the scope strategy the corresponding code looks like:

\begin{quote}
\begin{verbatim}
int fct()
{
  int a = 2;
  { int a = 3;
    int b = 4;
    return a+b;
  }
}
\end{verbatim}
\end{quote}

An exception of this approach is the translation of local function
definition in e.g.\ a {\em let expression}. In this case it is not
possible to use the C++ scope to model the restrictive scope of the
local function definition as C++ does not support local function
declarations in function declarations. See sections \ref{fctdef} and
\ref{expr} for a detailed description of the solution to this problem.


\subsubsection{Temporary Variables}
\label{tmpvar:sec}

The example above also raises another question, namely, the use of
temporary variables. In the example above the result of the function
is returned inside the inner block. However, in some way we have to
deal with the problem of when to return the result of an evaluation or
when the result of the expression is be used in an outer block.

In this second version of the specification we have chosen to reduce
the use of temporary variables compared to the previous version.  Some
expressions are allowed to return as C++ expressions(type CPP`Expr),
that is, the code for these expressions are not automatically assigned
to a temporary variable. Expressions which possibly can return as C++
expressions, are:

\begin{enumerate}
\item Prefix expressions.
\item Binary expressions.
\item Names.
\item Literals.
\item Function applications.
\item Bracketed expressions.
\end{enumerate}

As an example of this the specification

\begin{quote}
\begin{verbatim}
f:() -> nat
f() ==
  3 + 4
\end{verbatim}
\end{quote}

will have the corresponding code 
\begin{quote}
\begin{verbatim}
int f()
{
  return 3 + 4;
}
\end{verbatim}
\end{quote}

The corresponding code of the first example using this approach looks
like (since let expressions returns as C++ statements):

\begin{quote}
\begin{verbatim}
int fct()
{
  int a=2;
  int tmpRes;
  { int a = 3;
    int b = 4;
    tmpRes = a+b;
  }
  return tmpRes;
}
\end{verbatim}
\end{quote}

In this case the temporary variable {\tt tmpRes} is declared outside
the inner block, the result of the evaluation of the inner block is
assigned to the temporary variable and the result is returned in the
outer block. 

Another approach to reduce the number of temporary variables is in
function/operation definition.  If the arguments all are distinct
names, they can be declared directly in the argument list of the
function in the corresponding code. For instance, the specification

\begin{quote}
\begin{verbatim}
f:nat * nat * nat -> nat
f(a,b,c) ==
  a + b + c

\end{verbatim}
\end{quote}
has the code
\begin{quote}
\begin{verbatim}
int f(int a, int b, int c)
{
  return a + b + c;
}
\end{verbatim}
\end{quote}

whereas the specification

\begin{quote}
\begin{verbatim}
f:(nat * nat) * nat -> nat
f(mk_(a,b),c) ==
  a + b + c
\end{verbatim}
\end{quote}
will be coded like
\begin{quote}
\begin{verbatim}
int f( tuple var1, int var2)
{
  int a,b,c;
  ... // Pattern match
  return a + b + c;
}
\end{verbatim}
\end{quote}

\subsubsection{Type Information}\label{TI}

One of the ambition of the code generator is that it should be easy to
change the used basic data type implementation. In the first version
the \MCL{} is used as part of the basic data type implementation. The
\MCL{} is characterized by the fact that it offers the data type
$Generic$ which models the union type of all types. Another
characteristic of the \MCL{} is the construction of compound data
types (sets, sequences, maps, products and composites), a compound
data type can contain all types of values.  Facing these
characteristics of the \MCL{} it could be a temptation to use the data
type $Generic$ to model types which statically in the abstract
syntax tree are not obviously known. Consider the example below:

\begin{quote}
\begin{verbatim}
g: nat -> nat
g(b) ==
let a = f(b)
in h(a)
\end{verbatim}
\end{quote}

The data type of $a$ is not clear without looking up the signature
of the function $f$, thus using the \MCL{} a possible translation of
the let-expression could be:

\begin{quote}
\begin{verbatim}
Generic a = f(b);
return h(a);
\end{verbatim}
\end{quote}

However, as we want to design the code generator so that it is easy to
change the basic data type implementation and as we do not want to
require that a new data type implementation should provide this kind
of generic data types the point of view has been to infer the exact
data type in each declaration.  Consider the example above and assume
that the function $f$ has the signature:

\begin{quote}
\begin{verbatim}
f: nat -> set of ( nat | seq of nat )
\end{verbatim}
\end{quote}

The declaration of the variable $a$ should be the corresponding
declaration of the data type:

\begin{quote}
\begin{verbatim}
set of ( nat | seq of nat ) 
\end{verbatim}
\end{quote}

in the current data type refinement. Using the \MCL the corresponding
declaration is:

\begin{quote}
\begin{verbatim}
Set a = f(b);
\end{verbatim}
\end{quote}

The type information needed is supposed to be available in the
abstract syntax tree of the specification to be translated. See
section \ref{AS} for further information.

In this second version of the specification type information is added
to the abstract syntax tree for comprehensions and quantified
expressions.  This type information concerns the names in the
expressions. In the specification 

\begin{quote}
\begin{verbatim}
{a+b | a,[a,b] in set { 1,2,[2,1], [3,{6,7}],[{6,7},9] } } 
\end{verbatim}
\end{quote}

is the type for {\tt a} in {\tt a + b} integer. But the type for the
first {\tt a} in the bindlist is a union type of integer and sequence,
i.e. there exists an element in the set of the bindlist that never can
match {\tt a} due to its type. This knowledge is used to remove
never-matching elements from the set before the code for
the comprehension is generated. As it apperars, this will certanly not
reduce the size of the generated code(but the opposite), but the code
will be more efficient.  

\subsubsection{Pattern Match}

In one of the first proposals a more interpreter like approach was
described in the code generation of pattern match. In the code generating
pattern match a temporary variable was used which was assigned to a
map from identifiers names in the patterns to the values which the
pattern identifiers was bound to. Consider the example below:

\begin{quote}
\begin{verbatim}
cases {1, 2} :
  { a, b } -> mk_(a, b),
  others   -> 3
end
\end{verbatim}
\end{quote}

In the code corresponding to the pattern match of the set {\tt 1, 2}
against the pattern {\tt a, b} the temporary variable, say $bind$, was
e.g.\ assigned to: {\tt \{"a" |-> 2, "b" |-> 1\}}.  We have chosen not
to follow this strategy, one of the reasons is that this does not seem
to be a very efficient way. Instead we have chosen to ``fold out'' the
generated code, so that there is no need of internal variables, the
strategy chosen is to use the C++ scope to model binding of an
identifier to a value. See also section \ref{PM}.


\subsubsection{Library}

In this second version of the specification the code generator
auxillary library {\tt cg\_aux.cc } has been expanded with two
functions that was code generated in the previous version. This
strategy is an efficient way to reduce the size of the generated code
and should be investigated further in the versions to come.  The two
new functions are:

\begin{description}
\item[GenAllComb] This function is used by the comprehension
  expressions.
\item[Text2Seq ] This function inserts text literals from the abstract
  syntax into a sequence. 
\end{description}

\subsubsection{The Use of Union Types}

Consider the expression below:

\begin{quote}
\begin{verbatim}
let v1 = f(),
    v2 = g()
in v1 = v2
\end{verbatim}
\end{quote}

and assume that the signature of the functions are:

\begin{quote}
\begin{verbatim}
f: () -> nat | set of nat | map of nat to nat
g: () -> nat | set of nat
\end{verbatim}
\end{quote}

In the current strategy used in the code generator the declarations of
the variables corresponding to $v1$ and $v2$ correspond to the union
type appearing in the signatures of the functions $g$ and $f$. The
values assigned to the variables $v1$ and $v2$ are of the VDM union
type. The implementation of the union type is described in the module
$DS$ (see section \ref{DS}). One of the requirements to the
implementation of the VDM types is, for example, that it should be
possible to generate a C++ expression which compare any union value
with some other value (which does not necessary is of union type).
This restriction might not be easy to fulfill.


\subsubsection{Naming Conventions}
\label{renamevar}

In several cases a variable in the specification will correspond to a
variable in the generated code. The naming strategy used in the
code generator has been to rename all these variables to:
``vdm\_$module$\_$name$'', where $module$ is the name of the module in
which $name$ is defined. If the function $f$ e.g.\ is defined in
module $M$, then the C++ function corresponding to $f$ will be named
``vdm\_M\_f''. The case where the user has named a variable with the
name of a keyword in C++ does not raise a problem using this strategy.
Another advantage of this strategy is that the rest of the name space
for temporary variable names is available for the code generator. In
addition the following names are used by the code generator:

\begin{itemize}
\item init\_$module$: A function initializing values and the state of
  module $module$.
\item vdm\_$name$\_$number$: A temporary variable used by the code
  generator.
\item length\_$module$\_$record$: A macro defining the number of
  fields in the record $record$, defined in module $module$.
\item pos\_$module$\_$record$\_$field$: A macro defining the position
  (an integer) of the field selector $field$ in the record $record$,
  defined in module $module$.
\item vdm\_$module$\_implicit.cc: A file containing all user defined
  implementations of implicit functions and operations in module $module$.
\end{itemize}

Underscores (`\_') and single quotes (`'') appearing in variables in
the specification to be code generated will be exchanged with two
under scores (`\_\_') and underscore-single-quote (`\_q'),
respectively, in the generated C++ code.

Flat specifications will be transrormed into the module $DefaultMod$,
exporting all constructs (except the module state).



\subsubsection{Enumeration of Temporary Variables}

Consider the following VDM example of a set enumeration:

\begin{quote}
\begin{verbatim}
a = if b = 3 
    then { 1, 3, { 2, 3} }
    else 4
\end{verbatim}
\end{quote}

The type of $a$ will in this case be inferred to:

\begin{quote}
\begin{verbatim}
nat | set of ( nat | set of nat )
\end{verbatim}
\end{quote}

As it was explained in section \ref{tmpvar:sec} the variable
corresponding to $a$, {\tt vdm\_a}, is not usable when it comes to
evaluate the set enumeration. Thus, temporary variables have to be
introduced, However, we have to ensure that no name clashes appear in
a compound statement. Consider the pseudo code of the evaluation of
the set enumeration below (the pseudo code style is described in
section \ref{annospec:sec}):

\begin{quote}
\begin{verbatim}
Generic vdm_a;
{ Set tmpSetEnum                  \\ Temporary variable to contain the
                                  \\ result of the result of the
                                  \\ evaluation of "{ 1, 3, {2, 3} }"
  { tmpSetEnum.Insert( (Int) 1 ); \\ evaluation of "a" and insertion
                                  \\ in tmpSetEnum.
    tmpSetEnum.Insert( (Int) 3 ); 
    Set tmpSetEnum;               \\ Temporary variable to contain the
                                  \\ result of evaluation of "{2, 3}"
    .... Now, we have a name clash !!!!
  }
}
\end{verbatim}
\end{quote}

As it appears from the example we need to ensure that all temporary
variables are unique in a compound statement. We have chosen to
enumerate the temporary variables straightforward\footnote{In the
  examples of generated pseudo code the enumeration of the temporary
  variables has not been shown}. See also section \ref{BC}.


\subsection{Specification Style}

The specification is described as a combination of imperative an
functional parts. That is, operations and functions are used in
combination.

\subsubsection{Name Convention used in the specification of the Code Generator}

Following name convention has been used in the specification:
\begin{itemize}

\item All local variable names start with a lower case letter. An
  exception is when a variable clashes a key word. In this case the
  variable will start with an upper case letter.

\item For variables of type sequence, maps, sets, records, tuples and
  VarName the name might have a postfix being a underscore
  concatenated with the letter:
\begin{description}
\item[sequence: ] l.
\item[map: ] m.
\item[set: ] s.
\item[record: ] r.
\item[tuple: ] t.
\item[VarName: ] v.
\item[CPPAS: ] c
\end{description}
Underscores do not appear in variable names otherwise.

\end{itemize}


\subsubsection{Annotations to the specifications} 
\label{annospec:sec}

The specifications are annotated with VDM examples and corresponding
examples of the corresponding ``pseudo code'' generated by the current
specification. 

Basically, the idea behind the ``pseudo code'' is to give the reader
the overall structure/idea of the code generated by the specification
in question. We think that the understanding of the meaning of the
pseudo code examples are almost straightforward. However, some remarks
on the understanding of some of the $features$ which is used in
the pseudo language is listed below.

An example of a cases expression:
\begin{quote}
\begin{verbatim}
cases e:
  { a, b, 1 } -> b1,
  [1,b]       -> b2,
  others -> e 
end 
\end{verbatim} 
\end{quote}

where {\bf b{\tt n}} and {\tt e} denotes some expressions, that is,
these identifiers denote a kind of high-level macros in the VDM
example.


An example of ``pseudo code'' corresponding to the cases expression is
listed below:

\begin{small}
\begin{verbatim}
1  Generic tmpE;                // Declaration of a temporary variable to store 
2                               // the result of the evaluation of the expression e
3  ...CGExpr(e, tmpE);          // Code corresponding to the call CGGenExpr(e, tmpE)
4  int succ = FALSE;            // boolean indicating if a pattern match has succeeded
5
6  //Pattern match of first branch:
7  {
8    Generic vdm_a;             // Declaration of variables corresponding to
9    Generic vdm_b;             // the identifiers appearing in the pattern.
10  ...GCPatternMatch( "{a, b, 1}", tmpE, {}, succ);
                                // Code corresponding to the call 
11                              // of CGPatternMatch("{a, b, 1}", tmpE, {}, succ )
12  if ( succ )                 // if the pattern match of "{a,b,1}" against the value of e
13                              // has succeeded
14    {
15      ...GCExpr(b1, resVar)   // Code corresponding to the evaluation of the expression b1,
16                              // the identifier resVar is the variable in which the 
17                              // result of the evaluation of the expression b1 should be stored.
18    }
19 }
20
21 if ( ! succ )                // No pattern match succeeded in the first guard
22  {
23    Generic vdm_b;            // Declaration of the variable corresponding to 
24                              // the identifier appearing in the pattern.
25    ...CGPatternMatch( "[1,b]", tmpE, {}, succ ) 
                                // Code corresponding to the call
26                              // of CGPatternMatch("[1,b]", tmpE, {}, succ )
27  }
28 if ( !succ )                 // Not pattern match succeeded, evaluate the others branch.
29  {
30    ...CGExpr(e, resVar )     // Code corresponding to the evaluation of the expression e.
32  }
\end{verbatim}
\end{small}

Normally, no line numbers will appear in the pseudo code.  In line 3,
10, 15, 25 and 30 examples of a kind of macro call is shown. Three
dots concatenated by some function call means that at the current
place code corresponding to the code generated by calling the
operation or function should be replaced.  In line 10 and 25 the
sequences: {\tt "\{a, b, 1\}"} and {\tt "[1,b]"} appear, actually
these are shorthands for the abstract syntax tree of the corresponding
VDM patterns.

In most of the pseudo codes we have used \MCL{}.
#endif // VDMSL
#ifdef VDMPP
\section{Introduction}

This document contains the first draft of the specification
of the VDM++ part of the code generator front-end. In addition
it describes the overall structure of the specification and
the used strategies. 

The specification in this document supports class definitions
containing the following constructs:

\begin{itemize}
\item Inheritance:
  \begin{itemize}
  \item Representation.
  \item Indexed.
  \item Behavioral.
  \end{itemize}
\item Instance Variables ({\em invariants are not supported}).
\item Explicit member function definitions.
\item Value definitions.
\item Full and preliminary method specifications.
\item Statements:
  \begin{itemize}
  \item Invoke statements.
  \item New statement.
  \item Specification and Topology statements.
  \end{itemize}
\item Self Expressions
 \end{itemize}



\section{Overall Structure}

The specification is divided into the following modules:

\begin{description}
\item[CLASS:] This module provides functions generating code
  corresponding to VDM++ class definitions.

\item[FVD:] This module provides functions generating code
  corresponding to function and value definitions inside VDM++ class
  definitions.

\item[MD:] This module provides functions generating code
  corresponding to method definitions, methods modeling behavioral
  inheritance, and argument and type declarations/definitions.

\item[STMT:] This module is defined [IFAD-CG-4]\footnote{This is a
    doc. id.} bud is extended with functions generating code
  corresponding to VDM++ statements. That is, {\em Call Statement\/}
  has been changed into {\em Invoke Statement\/} and {\em New
    Statement\/}, and where {\em Specification} and {\em Topology
    Statements} have been added.

\item[EXPR:] This module is defined [IFAD-CG-4] and is extended with
  functions generation code corresponding to VDM++ expressions. That
  is, {\em isOfBaseClass Expression},
  {\em isOfClass Expression} and {\em Self Expression\/}. Note that
  only {\em Self Expression\/} is supported in the first version of
  the code generator.

\item[TI:] This module provides operations keeping track of type
  information collected by the type checker. This type information is
  used to determine methods class membership and dependencies between
  classes.

\end{description}

In addition to these modules, the specification uses of the modules
$CPP$, $BC$, $AUX$, $DS$, $PM$, $VD$ and $VD$. These are all described
in [IFAD-CG-4]. The VDM++ code generator uses the abstract syntax
(module $AS$) and the type representation (module $REP$) described in
\cite{Lassen&93a}.


Strategies, such as temporary variables, renaming, specification
style etc.\ used in this document will follow the same guidelines as
described in [IFAD-CG-4]. Classes will be renamed using
the following strings:

\begin{itemize}
\item {\em vdm\_\/} --- classes corresponding to VDM++ classes.

\item {\em iclass\_index\_super\_\/} --- classes modeling indexed
  inheritance from class {\em super\/} with the index {\em index\/}
  (see section \ref{sec:class}).

\end{itemize}


%\subsection{The Meta-IV C++ Library}
%
%
%The first version of the code generator will use the Meta-IV C++
%Library. This raises two problems in the VDM++ part of the code
%generator.
%
%In addition to the VDM-SL type set, VDM++ includes an object type,
%which can be combined with the other types in the usual way. This
%implies that the generated code can contain non Meta-IV objects which
%might be inserted in e.g.\ a Meta-IV sequence. Unfortunately this is
%not possible with out extending the Meta-IV library. The specification
%in this document do not concern with this problem, and is to be
%modified when the consequences to the code are clarified.
%
%Second, the way that indexed inheritance is modeled in the generated
%code can not be compiled with version 2.2.2 of the GNU C++ compiler
%(due to a bug). This bug has been fix in 2.5.8. The code generator
%generates code which is supported by  version 2.5.8 of the GNU C++
%compiler, and the Meta-IV library is to be ported to this version in
%order to be used with the generated code.
%
%The remaining part of this document will not be concerned with the
%former problems. This implies that parts of the specification are to
%modified then the Meta-IV library is extended and ported to GNU C++
%version 2.5.8.
%

\subsection{Run-Time Type Information}

In order to code generate the expressions {\em isOfBaseClass} and {\em
  isOfClass}, some run-time type information most be available (C++
does not provide a standard way of doing such type inquires). However,
the minimal information that a C++ class corresponding to a VDM++
class need, are a unique identification of the class (e.g.\ the class
name) and a unique identification of the root base classes (e.g.\ 
their names). The information can be stored in a member
variable in the class. The first version of the code generator will
not support {\em isOfBaseClass} and {\em isOfClass} and run-time type
information will therefore not be available for C++ classes generated
by the specification in this document.

Guidelines for adding run-time type information are described in
\cite{Stroustrup91}. 
#endif // VDMPP

#ifdef VDMSL
\section{Main Module CG}\label{CG}
\input{mod_cgmain.vdmsl.tex}

\section{Modules}
\input{mod_mod.vdm.tex}

\section{Function and Operation Definitions}
\label{fctdef}
\input{mod_fctdef.vdmsl.tex}


\section{Expressions}\label{expr}
\input{mod_expr.vdmsl.tex}


\section{Pattern Match}\label{PM}
\input{mod_patmat.vdm.tex}


\section{Statements}
\input{mod_stmt.vdmsl.tex}

\section{State Definitions}
\input{mod_statedef.vdm.tex}

\section{Type Definitions}\label{TD}
\input{mod_typedef.vdmsl.tex}

\section{Type Generation}\label{TPGEN}
\input{mod_tpgen.vdmsl.tex}

\section{Value Definitions}
\input{mod_valdef.vdmsl.tex}


\appendix



%\section{The Abstract Syntax of VDM}\label{AS}
%%\input{asslt.vdm.tex}


%\section{The Abstract Syntax of C++}
%\input{mod_cppast.vdm.tex}


\section{Module BC}

This module provides functions for building the abstract syntax tree
of C++.

\input{mod_bcppast.vdmsl.tex}


\section{Module DS}
\label{DS}
\input{mod_vdm_ds.vdmsl.tex}


\section{Module AUX}
\label{AUX}
\input{mod_cgaux.vdmsl.tex}
#endif // VDMSL

#ifdef VDMPP
%%%Specification.
\input{mod_cgmain.vdmpp.tex}

%%module CLASS
\input{mod_class.vdm.tex}

%%module FVD
\input{mod_fvd.vdm.tex}

\section{Operation and Function Definitions}\label{fctdef}
\input{mod_fctdef.vdmpp.tex}


\section{Expressions}\label{expr}
\input{mod_expr.vdmpp.tex}


\section{Pattern Match}\label{PM}
\input{mod_patmat.vdm.tex}


\section{Statements}
\input{mod_stmt.vdmpp.tex}

\section{Type Definitions}
\label{TD}
\input{mod_typedef.vdmpp.tex}

\section{Type Generation}\label{TPGEN}
\input{mod_tpgen.vdmpp.tex}


\section{Value Definitions}
\input{mod_valdef.vdmpp.tex}

#ifdef VDMPP
\section{Concurrency Constructs}

\input{mod_conc.vdm.tex}
#endif VDMPP

\appendix

%\section{The Definition of the Abstract Syntax of VDM}
%\label{AS}
%\input{common_as.vdmpp.tex}

%\label{AS}
%\input{rep.vdmpp.tex}

%\section{The Definition of the Abstract Syntax of C++}
#ifdef VDMSL
\input{mod_cppast.vdmsl.tex}
#elseif
\input{mod_cppast.vdmpp.tex}
#endif

%\section{Building the Abstract Syntax Tree of C++}
%This module provides functions for building the abstract syntax tree of C++. 
\input{mod_bcppast.vdmpp.tex}

%\section{Code Corresponding to Functions on VDM Data Structures}\label{DS}
\input{mod_vdm_ds.vdmpp.tex}


%\section{Auxiliary Module}\label{AUX}
\input{mod_cgaux.vdmpp.tex}

%%%module TI
\input{mod_ti.vdm.tex}
#endif // VDMPP


\bibliographystyle{iptes}
\bibliography{/home/peter/bib/dan}

\newpage
\addcontentsline{toc}{section}{Index}
\printindex

\end{document}



