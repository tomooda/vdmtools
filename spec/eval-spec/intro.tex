\chapter{Introduction}
\label{ch:intro}
\pagenumbering{arabic}
\raggedbottom

In~\cite{Bruin93a}, the specification and implementation of an interpreter
for the VDM-SL specification language is described. Because the
interpreter is a commercial product, we could not include the complete
specification as an appendix in the thesis. The document you are reading is
only available to the VDMTools Group and the Overture Tools Group.

In this report, we describe an abstract state machine semantics for the
VDM-SL language. This language supports both the complete ISO/VDM-SL
language (see~\cite{BSIVDM92}), and a structuring mechanism developed at
IFAD (see~\cite{Bruin93a}). In this report, the following is described:

\begin{itemize}
\item The complete abstract syntax of the VDM-SL language.
\item The semantic value domain used in the evaluation of a specification
  in VDM-SL.
\item The global definitions, types and state used in the evaluation.
\item The complete dynamic semantics for the language.
\end{itemize}

All parts are specified in the VDM-SL language. Therefore, some
syntactical differences must be taken into account when reading the
specifications. For a overview of the differences, see
also~\cite{Bruin93a}.

This report is based on an existing document, written also at IFAD. This
document, \cite{IPTESIFAD12}, describes an abstract state machine semantics
for the IPTES Meta-IV subset of ISO/VDM-SL. Therefore, not all parts of the
operation semantics described in this document are the result of the
MSc.~project. The following are additions/changes to the semantics
described in~\cite{IPTESIFAD12}:

\begin{itemize}
\item In~\cite{IPTESIFAD137}, a list of constructs is described that is not
  part of the old version of the VDM-SL language. The semantics of
  these constructs are added to the old operation semantics. These
  constructs include the support for higher order functions, local function
  definitions, lambda expressions, and the use of exception handling.

\item Due to some changes in the abstract syntax, some of the old
  specification had to be changed to comply to the new abstract syntax.

\item The complete specification of the structuring mechanism is added to
  the semantics.
\end{itemize}

% Local Variables: 
% mode: LaTeX
% TeX-master: "dynsem"
% End: 
