\documentclass[a4paper,dvips]{article}
\usepackage[dvips]{color}
\usepackage{float}
\usepackage{epsf}
\usepackage{vdmsl-2e}
\usepackage{longtable}
\usepackage{alltt}
\usepackage{makeidx}
\usepackage{ifad}

\definecolor{covered}{rgb}{0,0,0}      %black
%\definecolor{not_covered}{gray}{0.5}   %gray for previewing
\definecolor{not_covered}{gray}{0.6}   %gray for printing
%\definecolor{not_covered}{rgb}{1,0,0}  %red

\floatstyle{plain}
\restylefloat{figure}

\def\insertfig#1#2#3#4{ % Filename, epsfxsize, caption,  label
\begin{figure}[H]
\begin{center}
\leavevmode
\epsfxsize=#2
\epsffile{#1}
\end{center}
\caption{#3} #4 
\end{figure}
}

\newcommand{\StateDef}[1]{{\bf #1}}
\newcommand{\TypeDef}[1]{{\bf #1}}
\newcommand{\TypeOcc}[1]{{\it #1}}
\newcommand{\FuncDef}[1]{{\bf #1}}
\newcommand{\FuncOcc}[1]{#1}
\newcommand{\NonFuncOcc}[1]{}
\newcommand{\ModDef}[1]{}

\newcommand{\vdmslpp}[2]{%
#ifdef VDMSL
#1
#endif VDMSL
#ifdef VDMPP
#2
#endif VDMPP
}

\def\vdmsl{VDM-SL}
\def\vdmpp{VDM$^{++}$}

#ifdef VDMSL
\def\slpp{\vdmsl}
\def\otherslpp{\vdmpp}
#endif VDMSL
#ifdef VDMPP
\def\slpp{\vdmpp}
\def\otherslpp{\vdmsl}
#endif VDMPP

\makeindex
\begin{document}

\special{!userdict begin /bop-hook{gsave 220 30 translate
0 rotate /Times-Roman findfont 21 scalefont setfont
0 0 moveto (CONFIDENTIAL) show grestore}def end}

\docdef{Specification of the \slpp\ Static Semantics Checker}
{VDMTools Group}
{September, 2005}
#ifdef VDMSL
{VDMSL-3.8.7}
#endif VDMSL
#ifdef VDMPP
{VDM++ 6.8.7}
#endif VDMPP
{Technical}
{Intermediate}
{Confidential}
{}
{\copyright\ Kyushu University}
{\item[V1.0] First version
 \item[V1.1] Exception handling and other parts of VDM-SL which were
             not treated before have now been taken into account.
 \item[V1.2] Implementation modules are now taken into
   account. Well-formedness of parameterised modules has been
   tightened up and some bugs have been fixed.
 \item[V1.3] Prepared for December 1996 release.
} {}
\renewcommand{\thepage}{\roman{page}}

\tableofcontents
\newpage
\mbox{}
\newpage
\renewcommand{\thepage}{\arabic{page}}
\setcounter{page}{1}

\section{Introduction}

This document contains a specification of the static semantics of 
CSK \slpp. The specification is
written in CSK \vdmsl\ which corresponds to standard \vdmsl\
 \cite{BSIVDM92b} with a simple modular
extension as it is defined in \cite{deBruin93}\footnote{In this
document only simple imports (with renaming of constructs) and exports
are used.}. The specification is written in the executable
subset of VDM-SL which is supported by an interpreter (and debugger)
from the CSK VDM-SL Toolbox. Thus, this entire document is used in a
{\tt web} like fashion (see e.g.\ \cite{Smith&91}). This also means
that the specification (or to be more precise this
document) have been tested with a test suite with approximately 1000
test cases\footnote{In addition, the corresponding implementation of
the \otherslpp{} type checker have been tested with the same test suite.}.
The test suite itself (or the test module) is not included in this
document, but interested readers can get more information about it
from the authors.

This document is ordered according to the modular ordering which have
been chosen for the specification of the static semantics of \slpp.
However, prior to the actual modules this introduction is followed by
a section which gives an overview
of the module structure. The purpose of this section is to
introduce the chosen structuring and to enable the reader to find the
appropriate module for answering any specific questions.

\section{Module structure}

In this section we provide an overview of the structure
of the modules which are used in the static semantics for \slpp.

A sketch of the module structure and their relationships are presented
in figure~\ref{overview}. Parts of this figure will be used in the
explanation of the individual modules in the following sections. In
this section we will briefly explain the symbols which have been used and the
structuring which have been chosen.

All boxes indicates modules; the dotted module (AS)
is not present in this document. AS contains the abstract syntax for
\slpp.
The arcs between the boxes
indicates dependencies between the modules (in form of imports). A few
minor dependencies which do not help in the understanding of the
structuring have been omitted from figure~\ref{overview}, but these
minor dependencies are explicitly mentioned in the text.

Even though that there is a natural hierarchical order which is used
in the figure, it can be seen that mutual dependencies between the
modules occur. The three modules (AS, REP, and ERR) in the
bottom of the figure are imported by all the other modules (except
that ERR is not imported by the TYPE module).
The module in the margin VCM
is not considered to belong to the hierarchy at all because it
simply is used to indicate the interface to
the Version Control Manager. Thus, this module is not a part of the
static semantics.

\insertfig{overview.ps}{13cm}{Module overview}{\label{overview}} 

In the following we will explain the overall role of each individual
module in the hierarchical order.

\begin{description}
\item[DEF:] The main module checks the well-formedness of definitions
     which can appear at top-level (i.e.\ in the definition of a class).
     The DEF module simply exports its operation $wf-Type$ to the STMT
     module and the PAT module (a minor dependency) to check the
     well-formedness of locally used types. DEF mainly imports
     constructs from the modules at the second level of the hierarchy
     (STMT, EXPR and PAT), but it also uses a few constructs from the
     third level modules TYPE and ENV.
\item[STMT:] The statement module checks the well-formedness of
     statements which can be used as body of method definitions. The
     STMT module exports its top-level operation $wf-Stmt$ to
     the DEF module. A minor dependency is also
     present because $wf-Stmt$ is used by a single operation in the AUX
     module. STMT imports constructs from the other second level
     modules (EXPR and PAT), but it also import constructs from all
     third level modules (TYPE, AUX and ENV).
\item[EXPR:] The expression module checks the well-formedness of
     expressions. The EXPR module exports its main operation $wf-Expr$ (and
     a few other minor other constructs) to the DEF module and the STMT
     module (a minor dependency is also present
     because a pattern can contain an expression and thus one
     operation in PAT uses $wf-Expr$). EXPR imports constructs from
     PAT and all the third level modules (TYPE, AUX and ENV).
\item[PAT:] The pattern module checks the well-formedness of patterns
     and bindings. The PAT module exports its constructs to the DEF module
     and the other second level modules (STMT and EXPR). PAT imports
     constructs from all the third level modules (TYPE, AUX and
     ENV). As explained above there is also a minor dependency to the EXPR
     module.
\item[TYPE:] The well-formedness classification module is called TYPE
     because it is responsible for whether types are compatible with
     each other depending upon whether possibly or definitely
     well-formedness is being checked. No error messages are given
     from this module's operations. If two types are not compatible
     the calling operation will always be responsible for taking an
     appropriate action. The TYPE module exports its main operation
     $IsCompatible$ and the indication type $Ind$ (and a few other
     operations used in AUX) to the DEF module, and all the second
     level modules (STMT, EXPR and PAT) and to AUX. TYPE 
     imports a few constructs from the other third level modules.
\item[AUX:] This module simply contains a large number of auxiliary
     functions. These functions are mainly
     responsible for composing a collection of types
     into a new type according to the specific rules which apply to
     the specific construct. Thus, the functions in the AUX module are
     responsible for manipulating the inferred types to derive new
     inferred types depending upon the context in which they are used.
     The AUX module exports its constructs to all the second level
     modules (STMT, EXPR and PAT) and to the TYPE module. AUX
     import constructs from its other level three modules (TYPE and
     ENV). 
\item[ENV:] The environment module is responsible for maintaining the
     state of the static semantics (except for the error messages).
     It mainly consists of operations to lookup identifiers, scope
     manipulating operations and operations which expand the
     definitions to initialize the state. The ENV module exports its
     constructs to all the other modules from level 1 to 3 in the
     hierarchy. ENV only imports constructs from the VCM module which
     is not considered to belong to the hierarchy at all.   
\item[REP:] The type representation module contains the type
     definitions for the semantic domains which are used internally in
     the static semantics to model the types from the abstract syntax
     and the derived types. As mentioned above REP is imported by all
     modules in the hierarchy and it does only import the AS module.
\item[ERR:] The error module is responsible for keeping track of the
     errors which are found during a static semantics check. It does
     not import any constructs, but it is used in all modules in the
     hierarchy except the TYPE module. 
\item[VCM:] The Version Control Manager module is responsible for
     illustrating the functionality which the static semantics needs
     from the VCM tool. Thus, this is not a part of the static
     semantics, but it indicates the interface which is needed from
     the static semantics. VCM is only used in the ENV module and it
     only imports the AS and the REP modules.
\end{description}

#ifdef VDMSL
\input{def.vdmsl.tex}
\input{stmt.vdmsl.tex}
\input{expr.vdmsl.tex}
\input{pat.vdmsl.tex}
\input{type.vdmsl.tex}
\input{tcaux.vdmsl.tex}
\input{env.vdmsl.tex}
\input{rep.vdmsl.tex}
\input{err.vdmsl.tex}
\input{vcm.vdmsl.tex}
#endif VDMSL
#ifdef VDMPP
#ifdef VICE
\input{def.vice.tex}
\input{stmt.vice.tex}
\input{expr.vice.tex}
\input{pat.vice.tex}
\input{type.vice.tex}
\input{tcaux.vice.tex}
\input{env.vice.tex}
\input{rep.vice.tex}
\input{err.vice.tex}
\input{vcm.vice.tex}
#else
\input{def.vdmpp.tex}
\input{stmt.vdmpp.tex}
\input{expr.vdmpp.tex}
\input{pat.vdmpp.tex}
\input{type.vdmpp.tex}
\input{tcaux.vdmpp.tex}
\input{env.vdmpp.tex}
\input{rep.vdmpp.tex}
\input{err.vdmpp.tex}
\input{vcm.vdmpp.tex}
#endif VICE
#endif VDMPP

\newpage

\bibliographystyle{newalpha}
\bibliography{/Users/sahara/CVSCheckout/toolbox/doc/bib/ifad}

\newpage
\addcontentsline{toc}{section}{Index}
\printindex

\end{document}

